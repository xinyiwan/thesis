%================================================================
%================================================================
Diffusion magnetic resonance imaging(dMRI) is a powerful imaging modality 
capable of inferring the local axonal structure in each imaging voxel 
by exploiting the natural random movement of water molecules 
in biological tissues \cite{intro}. Tractography is a set of algorithms aiming at 
mapping the major neuronal pathways in the white matter of the brain from dMRI signals. 
Based on the estimated direction of neuronal fibres from dMRI data, the tractogram of 
the brain could be generated with tractography, which consists of numerous streamlines that 
represent the microstructure of brain tissues. 
According to its technical features, the tractography of dMRI could contribute to 
the noninvasive investigation of brain connectivity and have potential in clinical research. 

However, there are still concerns and challenges with tractography. 
The anatomically implausible streamlines from tractography might influence 
research results from tractogram, such as the analysis of brain connectivity. 
To solve this problem, tractogram filtering methods have been developed to 
remove faulty connections in a post-processing step \cite{hain2022assessing}.

This project mainly analyses one of the tractogram filtering tools,
\textit{Convex Optimization Modeling for Microstructure Informed Tractography}(COMMIT), 
which aims at reestablishing the link between tractography and tissue microstructure \cite{daducci2014commit}. 
First, the project aims to investigate the performance of COMMIT. 
It is meant to explore the results from COMMIT when it faces streamlines with 
different geometrical features. A dataset from ISMRA challenge\cite{maier_hein_klaus_2015_572345} 
based on the ground-truth bundles will be used to generate different inputs of 
COMMIT with known truth-positive, redundant, and false-positive streamlines.

Another problem with filtering methods is the inconsistent results from COMMIT, 
which are sensitive to the size and composition of the surrounding tractogram. 
Although the anatomical ground truth is absent, a pseudo ground truth could be 
generalized from the partly consistent results of COMMIT. By applying COMMIT to 
multiple tractogram subsets, consistent assessments for each streamline are extracted
 as pseudo-ground truths. A deep-learning-based classifier trained by pseudo-ground truths
  will be implemented to enhance the results to improve the plausibility of streamlines.

In the end, the classifier will be validated with connectivity maps between a healthy group and a group of brain diseases to compare the connectivity difference of both groups before and after using randomised COMMIT.

% One can use either biblatex or bibtex - set as the option for the document at the top of this file
\ifbiblatex
\engExpl{We use the \emph{biblatex} package to handle our references.  We
use the command \texttt{parencite} to get a reference in parenthesis, like
this \textbackslash parencite\{heisenberg2015\} resulting in \parencite{heisenberg2015}.  It is also possible to include the author as part of the sentence using \texttt{textcite}, like talking about the work of \textbackslash textcite\{einstein2016\} resulting in \textcite{einstein2016}.\\
This also means that you have to change the include files to include biblatex and change the way that the reference.bib file is included.}
\else
\engExpl{We use the \emph{bibtex} package to handle our references.  We therefore
use the command \textbackslash cite\{farshin\_make\_2019\}. For example, Farshin, \etal described how to improve LLC
cache performance in \cite{farshin_make_2019} in the context of links running
at \qty{200}{Gbps}.}
\fi

\engExpl{Use the glossaries package to help yourself and your readers.
Add the acronyms and abbreviations to lib/acronyms.tex. Some examples are shown below:}
In this thesis we will examine the use of \glspl{LAN}. In this thesis we will
assume that \glspl{LAN} include \glspl{WLAN}, such as \gls{WiFi}.


\section{Background}
\label{sec:background}

%================================================================
%================================================================









\sweExpl{svensk: Bakgrund}

\generalExpl{Present the background for the area. Set the context for your project – so that your reader can understand both your project and this thesis. (Give detailed background information in Chapter 2 - together with related work.)
Sometimes it is useful to insert a system diagram here so that the reader
knows what are the different elements and their relationship to each
other. This also introduces the names/terms/… that you are going to use
throughout your thesis (be consistent). This figure will also help you later
delimit what you are going to do and what others have done or will do.}

As one can find in RFC 1235\,\cite{ioannidis_coherent_1991} multicast is useful for xxxx. A number of different \glspl{OS} have been used in this work, such as the following \glspl{OS}: UNIX, Linux, Windows, etc. The main focus will be on one \gls{OS}, namely Linux.

\section{Problem}
\label{sec:problem}
\sweExpl{svensk: Problemdefinition eller Frågeställning\\
Lyft fram det ursprungliga problemet om det finns något och definiera därefter
den ingenjörsmässiga erfarenheten eller/och vetenskapen som kan komma ur
projektet. }

Longer problem statement\\
If possible, end this section with a question as a problem statement.

% Research Question
\subsection{Original problem and definition}
\label{sec:researchQuestion}

\subsection{Scientific and engineering issues}
