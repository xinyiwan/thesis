
%%
%% forked from https://gits-15.sys.kth.se/giampi/kthlatex kthlatex-0.2rc4 on 2020-02-13
%% expanded upon by Gerald Q. Maguire Jr.
%% This template has been adapted by Anders Sjögren to the University
%% Engineering Program in Computer Science at KTH ICT. Adaptation is the
%% translation of English headings into Swedish as the addition of Swedish


%% The template is designed to handle a thesis in English or Swedish
% set the default language to english or swedish by passing an option to the documentclass - this handles the inside tile page
% To optimize for digital output (this changes the color palette add the option: digitaloutput
% To use \ifnomenclature add the option nomenclature
% To use bibtex or biblatex - include one of these as an option
\documentclass[nomenclature, english, bibtex]{kththesis}
%\documentclass[swedish, biblatex]{kththesis}
% if pdflatex \usepackage[utf8]{inputenc}

%% Conventions for todo notes:
% Informational
%% \generalExpl{Comments/directions/... in English}
\newcommand*{\generalExpl}[1]{\todo[inline]{#1}}                

% Language specific information (currently in English or Swedish)
\newcommand*{\engExpl}[1]{\todo[inline, backgroundcolor=kth-lightgreen40]{#1}} %% \engExpl{English descriptions about formatting}
\newcommand*{\sweExpl}[1]{\todo[inline, backgroundcolor=kth-lightblue40]{#1}}  %% % \sweExpl{Text på svenska}

% warnings
\newcommand*{\warningExpl}[1]{\todo[inline, backgroundcolor=kth-lightred40]{#1}} %% \warningExpl{warnings}

% Uncomment to hide specific comments, to hide **all** ToDos add `final` to
% document class
% \renewcommand\warningExpl[1]{}
% \renewcommand\generalExpl[1]{}
% \renewcommand\engExpl[1]{}
% For example uncommenting the following line hides the Swedish language explanations
% \renewcommand\sweExpl[1]{}


% \usepackage[style=numeric,sorting=none,backend=biber]{biblatex}
\ifbiblatex
    %\usepackage[language=english,bibstyle=authoryear,citestyle=authoryear, maxbibnames=99]{biblatex}
    % alternatively you might use another style, such as IEEE and use citestyle=numeric-comp  to put multiple citations in a single pair of square brackets
    %\usepackage[style=ieee,citestyle=numeric-comp]{biblatex}
    \addbibresource{references.bib}
    %\DeclareLanguageMapping{norsk}{norwegian}
\else
    % The line(s) below are for BibTeX
    \bibliographystyle{bibstyle/myIEEEtran}
    %\bibliographystyle{apalike}
\fi


% include a variety of packages that are useful
\input{lib/includes}
\input{lib/kthcolors}

%\glsdisablehyper
%\makeglossaries
%\makenoidxglossaries
%\input{lib/acronyms}                %load the acronyms file

\input{lib/defines}  % load some additional definitions to make writing more consistent

% The following is needed in conjunction with generating the DiVA data with abstracts and keywords using the scontents package and a modified listings environment
%\usepackage{listings}   %  already included
\ExplSyntaxOn
\newcommand\typestoredx[2]{\expandafter\__scontents_typestored_internal:nn\expandafter{#1} {#2}}
\ExplSyntaxOff
\makeatletter
\let\verbatimsc\@undefined
\let\endverbatimsc\@undefined
\lst@AddToHook{Init}{\hyphenpenalty=50\relax}
\makeatother


\lstnewenvironment{verbatimsc}
    {
    \lstset{%
        basicstyle=\ttfamily\tiny,
        backgroundcolor=\color{white},
        %basicstyle=\tiny,
        %columns=fullflexible,
        columns=[l]fixed,
        language=[LaTeX]TeX,
        %numbers=left,
        %numberstyle=\tiny\color{gray},
        keywordstyle=\color{red},
        breaklines=true,                 % sets automatic line breaking
        breakatwhitespace=true,          % sets if automatic breaks should only happen at whitespace
        %keepspaces=false,
        breakindent=0em,
        %fancyvrb=true,
        frame=none,                     % turn off any box
        postbreak={}                    % turn off any hook arrow for continuation lines
    }
}{}

%% Add some more keyowrds to bring out the structure more
\lstdefinestyle{[LaTeX]TeX}{
morekeywords={begin, todo, textbf, textit, texttt}
}

%% definition of new command for bytefield package
\newcommand{\colorbitbox}[3]{%
	\rlap{\bitbox{#2}{\color{#1}\rule{\width}{\height}}}%
	\bitbox{#2}{#3}}




% define a left aligned table cell that is ragged right
\newcolumntype{L}[1]{>{\raggedright\let\newline\\\arraybackslash\hspace{0pt}}p{#1}}

% Because backref is not compatible with biblatex
\ifbiblatex
    \usepackage[plainpages=false]{hyperref}
\else
    \usepackage[
    backref=page,
    pagebackref=false,
    plainpages=false,
                            % PDF related options
    unicode=true,           % Unicode encoded PDF strings
    bookmarks=true,         % generate bookmarks in PDF files
    bookmarksopen=false,    % Do not automatically open the bookmarks in the PDF reading program
    pdfpagemode=UseNone,    % None, UseOutlines, UseThumbs, or FullScreen
    destlabel,              % better naming of destinations
    ]{hyperref}
    \usepackage{backref}
    %
    % Customize list of backreferences.
    % From https://tex.stackexchange.com/a/183735/1340
    \renewcommand*{\backref}[1]{}
    \renewcommand*{\backrefalt}[4]{%
    \ifcase #1%
          \or [Page~#2.]%
          \else [Pages~#2.]%
    \fi%
    }
\fi
\usepackage[all]{hypcap}	%% prevents an issue related to hyperref and caption linking

%% Acronyms
% note that nonumberlist - removes the cross references to the pages where the acronym appears
% note that super will set the descriptions text aligned
% note that nomain - does not produce a main glossary, thus only acronyms will be in the glossary
% note that nopostdot - will prevent there being a period at the end of each entry
\usepackage[acronym, style=super, section=section, nonumberlist, nomain,
nopostdot]{glossaries}
\setlength{\glsdescwidth}{0.75\textwidth}
\usepackage[]{glossaries-extra}
\ifinswedish
    %\usepackage{glossaries-swedish}
\fi

%% For use with the README_notes
% Define a new type of glossary so that the acronyms defined in the README_notes document can be distinct from those in the thesis template
% the tlg, tld, and dn will be the file extensions used for this glossary
\newglossary[tlg]{readme}{tld}{tdn}{README acronyms}


% packages that have to be included after hyperref
\usepackage{doi}
\usepackage{cleveref}           %% Replace Section with a symbol


%\glsdisablehyper
\makeglossaries
%\makenoidxglossaries
\input{lib/acronyms}                %load the acronyms file

% insert the configuration information with author(s), examiner, supervisor(s), ...
\input{custom_configuration}

\title{This is the title in the language of the thesis}
\subtitle{A subtitle in the language of the thesis}

% give the alternative title - i.e., if the thesis is in English, then give a Swedish title
\alttitle{Detta är den svenska översättningen av titeln}
\altsubtitle{Detta är den svenska översättningen av undertiteln}
% alternative, if the thesis is in Swedish, then give an English title
%\alttitle{This is the English translation of the title}
%\altsubtitle{This is the English translation of the subtitle}

% Enter the English and Swedish keywords here for use in the PDF meta data _and_ for later use
% following the respective abstract.
% Try to put the words in the same order in both languages to facilitate matching. For example:
\EnglishKeywords{Canvas Learning Management System, Docker containers, Performance tuning}
\SwedishKeywords{Canvas Lärplattform, Dockerbehållare, Prestandajustering}

%%%%% For the oral presentation
%% Add this information once your examiner has scheduled your oral presentation
\presentationDateAndTimeISO{2022-03-15 13:00}
\presentationLanguage{eng}
\presentationRoom{via Zoom https://kth-se.zoom.us/j/ddddddddddd}
\presentationAddress{Isafjordsgatan 22 (Kistagången 16)}
\presentationCity{Stockholm}

% When there are multiple opponents, separate their names with '\&'
% Opponent's information
\opponentsNames{A. B. Normal \& A. X. E. Normalè}

% Once a thesis is approved by the examiner, add the TRITA number
% The TRITA number for a thesis consists of two parts a series (unique to each school)
% and the number in the series which is formatted as the year followed by a colon and
% then a unique series number for the thesis - starting with 1 each year.
\trita{TRITA-EECS-EX}{2022:00}

% Put the title, author, and keyword information into the PDF meta information
\input{lib/pdf_related_includes}


% the custom colors and the commands are defined in defines.tex    
\hypersetup{
	colorlinks  = true,
	breaklinks  = true,
	linkcolor   = \linkscolor,
	urlcolor    = \urlscolor,
	citecolor   = \refscolor,
	anchorcolor = black
}

\ifnomenclature
% The following lines make the page numbers and equations hyperlinks in the Nomenclature list
\renewcommand*{\pagedeclaration}[1]{\unskip, \dotfill\hyperlink{page.#1}{page\nobreakspace#1}}
% The following does not work correctly, as the name of the cross-reference is incorrect
%\renewcommand*{\eqdeclaration}[1]{, see equation\nobreakspace(\hyperlink{equation.#1}{#1})}

% You can also change the page heading for the nomenclature
\renewcommand{\nomname}{List of Symbols Used}

% You can even add customization text before the list
\renewcommand{\nompreamble}{The following symbols will be later used within the body of the thesis.}
\makenomenclature
\fi

%
% The commands below are to configure JSON listings
% 
% format for JSON listings
\colorlet{punct}{red!60!black}
\definecolor{delim}{RGB}{20,105,176}
\definecolor{numb}{RGB}{106, 109, 32}
\definecolor{string}{RGB}{0, 0, 0}

\lstdefinelanguage{json}{
    numbers=none,
    numberstyle=\small,
    frame=none,
    rulecolor=\color{black},
    showspaces=false,
    showtabs=false,
    breaklines=true,
    postbreak=\raisebox{0ex}[0ex][0ex]{\ensuremath{\color{gray}\hookrightarrow\space}},
    breakatwhitespace=true,
    basicstyle=\ttfamily\small,
    extendedchars=false,
    upquote=true,
    morestring=[b]",
    stringstyle=\color{string},
    literate=
     *{0}{{{\color{numb}0}}}{1}
      {1}{{{\color{numb}1}}}{1}
      {2}{{{\color{numb}2}}}{1}
      {3}{{{\color{numb}3}}}{1}
      {4}{{{\color{numb}4}}}{1}
      {5}{{{\color{numb}5}}}{1}
      {6}{{{\color{numb}6}}}{1}
      {7}{{{\color{numb}7}}}{1}
      {8}{{{\color{numb}8}}}{1}
      {9}{{{\color{numb}9}}}{1}
      {:}{{{\color{punct}{:}}}}{1}
      {,}{{{\color{punct}{,}}}}{1}
      {\{}{{{\color{delim}{\{}}}}{1}
      {\}}{{{\color{delim}{\}}}}}{1}
      {[}{{{\color{delim}{[}}}}{1}
      {]}{{{\color{delim}{]}}}}{1}
      {’}{{\char13}}1,
}

\lstdefinelanguage{XML}
{
  basicstyle=\ttfamily\color{blue}\bfseries\small,
  morestring=[b]",
  morestring=[s]{>}{<},
  morecomment=[s]{<?}{?>},
  stringstyle=\color{black},
  identifierstyle=\color{blue},
  keywordstyle=\color{cyan},
  breaklines=true,
  postbreak=\raisebox{0ex}[0ex][0ex]{\ensuremath{\color{gray}\hookrightarrow\space}},
  breakatwhitespace=true,
  morekeywords={xmlns,version,type}% list your attributes here
}

% In case you use both listings and lstlistings - this makes them both use the same counter
\makeatletter
\AtBeginDocument{\let\c@listing\c@lstlisting}
\makeatother
\usepackage{subfiles}

% To have Creative Commons (CC) license and logos use the doclicense package
% Note that the lowercase version of the license has to be used in the modifier
% i.e., one of by, by-nc, by-nd, by-nc-nd, by-sa, by-nc-sa, zero.
% For background see:
% https://www.kb.se/samverkan-och-utveckling/oppen-tillgang-och-bibsamkonsortiet/open-access-and-bibsam-consortium/open-access/creative-commons-faq-for-researchers.html
% https://kib.ki.se/en/publish-analyse/publish-your-article-open-access/open-licence-your-publication-cc
\begin{comment}
\usepackage[
    type={CC},
    %modifier={by-nc-nd},
    %version={4.0},
    modifier={by-nc},
    imagemodifier={-eu-88x31},  % to get Euro symbol rather than Dollar sign
    hyphenation={RaggedRight},
    version={4.0},
    %modifier={zero},
    %version={1.0},
]{doclicense}
\end{comment}

% ====================================================================================================
% ====================================================================================================

\begin{document}
%\selectlanguage{swedish}
%
\selectlanguage{english}

%%% Set the numbering for the title page to a numbering series not in the preface or body
\pagenumbering{alph}
\kthcover
\clearpage\thispagestyle{empty}\mbox{} % empty back of front cover
\titlepage

% If you do not want to have a bookinfo page, comment out the line saying \bookinfopage and add a \cleardoublepage
% If you want a bookinfo page: you will get a copyright notice, unless you have used the doclicense package in which case you will get a Creative Commons license. To include the doclicense package, uncomment the configuration of this package above and configure it with your choice of license.
\bookinfopage

% Frontmatter includes the abstracts and table-of-contents
\frontmatter
\setcounter{page}{1}

\begin{abstract}
% The first abstract should be in the language of the thesis.
% Abstract fungerar på svenska också.
  \markboth{\abstractname}{}
\begin{scontents}[store-env=lang]
eng
\end{scontents}
%%% The contents of the abstract (between the begin and end of scontents) will be saved in LaTeX format
%%% and output on the page(s) at the end of the thesis with information for DiVA facilitating the correct
%%% entry of the meta data for your thesis.
%%% These page(s) will be removed before the thesis is inserted into DiVA.
\engExpl{All theses at KTH are \textbf{required} to have an abstract in both \textit{English} and \textit{Swedish}.}

\engExpl{Exchange students many want to include one or more abstracts in the language(s) used in their home institutions to avoid the need to write another thesis when returning to their home institution.}

\generalExpl{Keep in mind that most of your potential readers are only going to read your \texttt{title} and \texttt{abstract}. This is why it is important that the abstract give them enough information that they can decide is this document relevant to them or not. Otherwise the likely default choice is to ignore the rest of your document.\\
A abstract should stand on its own, i.e., no citations, cross references to the body of the document, acronyms must be spelled out, \ldots .\\Write this early and revise as necessary. This will help keep you focused on what you are trying to do.}
\begin{scontents}[store-env=abstracts,print-env=true]
\generalExpl{Enter your abstract here!}
Write an abstract that is about 250 and 350 words (1/2 A4-page)  with the following components:
% key parts of the abstract
\begin{itemize}
  \item What is the topic area? (optional) Introduces the subject area for the project.
  \item Short problem statement
  \item Why was this problem worth a Bachelor's/Master’s thesis project? (\ie, why is the problem both significant and of a suitable degree of difficulty for a Bachelor's/Master’s thesis project? Why has no one else solved it yet?)
  \item How did you solve the problem? What was your method/insight?
  \item Results/Conclusions/Consequences/Impact: What are your key results/\linebreak[4]conclusions? What will others do based upon your results? What can be done now that you have finished - that could not be done before your thesis project was completed?
\end{itemize}

\end{scontents}
\engExpl{The following are some notes about what can be included (in terms of LaTeX) in your abstract.}
Choice of typeface with \textbackslash textit, \textbackslash textbf, and \textbackslash texttt:  \textit{x}, \textbf{x}, and \texttt{x}.

Text superscripts and subscripts with \textbackslash textsubscript and \textbackslash textsuperscript: A\textsubscript{x} and A\textsuperscript{x}.

Some symbols that you might find useful are available, such as: \textbackslash textregistered, \textbackslash texttrademark, and \textbackslash textcopyright. For example, 
the copyright symbol: \textbackslash textcopyright Maguire 2022 results in \textcopyright Maguire 2022. Additionally, here are some examples of text superscripts (which can be combined with some symbols): \textbackslash textsuperscript\{99m\}Tc, A\textbackslash textsuperscript\{*\}, A\textbackslash textsuperscript\{\textbackslash textregistered\}, and A\textbackslash texttrademark resulting in \textsuperscript{99m}Tc, A\textsuperscript{*}, A\textsuperscript{\textregistered}, and A\texttrademark. Two examples of subscripts are: H\textbackslash textsubscript\{2\}O and CO\textbackslash textsubscript\{2\} which produce  H\textsubscript{2}O and CO\textsubscript{2}.

Simple environments with begin and end: itemize and enumerate and within these \textbackslash item.

The following commands can be used: \textbackslash eg, \textbackslash Eg, \textbackslash ie, \textbackslash Ie, \textbackslash etc, and \textbackslash etal: \eg, \Eg, \ie, \Ie, \etc, and \etal.

The following commands for numbering with lower case Roman numerals: \textbackslash first, \textbackslash Second, \textbackslash third, \textbackslash fourth, \textbackslash fifth, \textbackslash sixth, \textbackslash seventh, and \textbackslash eighth: \first, \Second, \third, \fourth, \fifth, \sixth, \seventh, and \eighth. Note that the second case is set with a capital 'S' to avoid conflicts with the use second of as a unit in the \texttt{siunitx} package.

Equations using \textbackslash( xxxx \textbackslash) or \textbackslash[ xxxx \textbackslash] can be used in the abstract. For example: \( (C_5O_2H_8)_n \)
or \[ \int_{a}^{b} x^2 \,dx \]
Note that you \textbf{cannot} use an equation between dollar signs.

Even LaTeX comments can be handled, for example: \% comment.
Note that one can include percentages, such as: 51\% or \SI{51}{\percent}.

\subsection*{Keywords}
\begin{scontents}[store-env=keywords,print-env=true]
% If you set the EnglishKeywords earlier, you can retrieve them with:
\InsertKeywords{english}
% If you did not set the EnglishKeywords earlier then simply enter the keywords here:
% comma separate keywords, such as: Canvas Learning Management System, Docker containers, Performance tuning
\end{scontents}
\engExpl{\textbf{Choosing good keywords can help others to locate your paper, thesis, dissertation, \ldots and related work.}}
Choose the most specific keyword from those used in your domain, see for example: the ACM Computing Classification System ({\small \url{https://www.acm.org/publications/computing-classification-system/how-to-use})},
the IEEE Taxonomy ({\small \url{https://www.ieee.org/publications/services/thesaurus-thank-you.html}}), PhySH (Physics Subject Headings)\linebreak[4] ({\small \url{https://physh.aps.org/}}), \ldots or keyword selection tools such as the  National Library of Medicine's Medical Subject Headings (MeSH)  ({\small \url{https://www.nlm.nih.gov/mesh/authors.html}}) or Google's Keyword Tool ({\small \url{https://keywordtool.io/}})\\
\clearpage
\textbf{Formatting the keywords}:
\begin{itemize}
  \item The first letter of a keyword should be set with a capital letter and proper names should be capitalized as usual.
  \item Spell out acronyms and abbreviations.
  \item Avoid "stop words" - as they generally carry little or no information.
  \item List your keywords separated by commas (",").
\end{itemize}    
Since you should have both English and Swedish keywords - you might think of ordering them in corresponding order (\ie, so that the n\textsuperscript{th} word in each list correspond) - this makes it easier to mechanically find matching keywords.
\end{abstract}
\cleardoublepage
\babelpolyLangStart{swedish}
\begin{abstract}
    \markboth{\abstractname}{}
\begin{scontents}[store-env=lang]
swe
\end{scontents}
\begin{scontents}[store-env=abstracts,print-env=true]
\generalExpl{Enter your Swedish abstract or summary here!}
\sweExpl{Alla avhandlingar vid KTH \textbf{måste ha} ett abstrakt på både \textit{engelska} och \textit{svenska}.\\
Om du skriver din avhandling på svenska ska detta göras först (och placera det som det första abstraktet) - och du bör revidera det vid behov.}

\engExpl{If you are writing your thesis in English, you can leave this until the draft version that goes to your opponent for the written opposition. In this way you can provide the English and Swedish abstract/summary information that can be used in the announcement for your oral presentation.\\If you are writing your thesis in English, then this section can be a summary targeted at a more general reader. However, if you are writing your thesis in Swedish, then the reverse is true – your abstract should be for your target audience, while an English summary can be written targeted at a more general audience.\\This means that the English abstract and Swedish sammnfattning  
or Swedish abstract and English summary need not be literal translations of each other.}

\warningExpl{Do not use the \textbackslash glspl\{\} command in an abstract that is not in English, as my programs do not know how to generate plurals in other languages. Instead you will need to spell these terms out or give the proper plural form. In fact, it is a good idea not to use the glossary commands at all in an abstract/summary in a language other than the language used in the \texttt{acronyms.tex file} - since the glossary package does \textbf{not} support use of more than one language.}

\engExpl{The abstract in the language used for the thesis should be the first abstract, while the Summary/Sammanfattning in the other language can follow}
\end{scontents}
\subsection*{Nyckelord}
\begin{scontents}[store-env=keywords,print-env=true]
% SwedishKeywords were set earlier, hence we can use alternative 2
\InsertKeywords{swedish}
\end{scontents}
\sweExpl{Nyckelord som beskriver innehållet i uppsatsen eller rapporten}
\end{abstract}
\babelpolyLangStop{swedish}

\cleardoublepage


\section*{Acknowledgments}
\markboth{Acknowledgments}{}
\sweExpl{Författarnas tack}

\engExpl{It is nice to acknowledge the people that have helped you. It is
  also necessary to acknowledge any special permissions that you have gotten –
  for example getting permission from the copyright owner to reproduce a
  figure. In this case you should acknowledge them and this permission here
  and in the figure’s caption. \\
  Note: If you do \textbf{not} have the copyright owner’s permission, then you \textbf{cannot} use any copyrighted figures/tables/\ldots . Unless stated otherwise all figures/tables/\ldots are generally copyrighted.
}
\sweExpl{I detta kapitel kan du ev nämna något om
  din bakgrund om det påverkar rapporten på något sätt. Har du t ex inte
  möjlighet att skriva perfekt svenska för att du är nyanländ till landet kan
  det vara på sin plats att nämna detta här. OBS, detta får dock inte vara en
  ursäkt för att lämna in en rapport med undermåligt språk, undermålig grammatik och
  stavning (t ex får fel som en automatisk stavningskontroll och
  grammatikkontroll kan upptäcka inte förekomma)\\
En dualism som måste hanteras i hela rapporten och projektet
}

I would like to thank xxxx for having yyyy. Or in the case of two authors:\\
We would like to thank xxxx for having yyyy.

\acknowlegmentssignature

\fancypagestyle{plain}{}
\renewcommand{\chaptermark}[1]{ \markboth{#1}{}} 
\tableofcontents
  \markboth{\contentsname}{}

\cleardoublepage
\listoffigures

\cleardoublepage

\listoftables
\cleardoublepage
\lstlistoflistings\engExpl{If you have listings in your thesis. If not, then remove this preface page.}
\cleardoublepage
% Align the text expansion of the glossary entries
\newglossarystyle{mylong}{%
  \setglossarystyle{long}%
  \renewenvironment{theglossary}%
     {\begin{longtable}[l]{@{}p{\dimexpr 2cm-\tabcolsep}p{0.8\hsize}}}% <-- change the value here
     {\end{longtable}}%
 }
%\glsaddall
%\printglossaries[type=\acronymtype, title={List of acronyms}]
\printglossary[style=mylong, type=\acronymtype, title={List of acronyms and abbreviations}]
%\printglossary[type=\acronymtype, title={List of acronyms and abbreviations}]

%\printnoidxglossary[style=mylong, title={List of acronyms and abbreviations}]
\engExpl{The list of acronyms and abbreviations should be in alphabetical order based on the spelling of the acronym or abbreviation.
}

% if the nomenclature option was specified, then include the nomenclature page(s)
\ifnomenclature
    \cleardoublepage
    % Output the nomenclature list
    \printnomenclature
\fi

%% The following label is essential to know the page number of the last page of the preface
%% It is used to computer the data for the "For DIVA" pages
\label{pg:lastPageofPreface}



% ====================================================================================================
% ====================================================================================================
% Mainmatter is where the actual contents of the thesis goes
\mainmatter
\glsresetall
\renewcommand{\chaptermark}[1]{\markboth{#1}{}}
\selectlanguage{english}
\chapter{Introduction}
\label{ch:introduction}
\sweExpl{svensk: Introduktion}


\sweExpl{Ofta kommer problemet och problemägaren från industrin där man önskar en specifik lösning på ett specifikt problem. Detta är ofta ”för smalt” definierat och ger ofta en ”för smal” lösning för att resultatet skall vara intressant ur ett mer allmänt ingenjörsperspektiv och med ”nya” erfarenheter som resultat. Fundera tillsammans med projektets intressenter (student, problemägare och akademi) hur man skulle kunna använda det aktuella problemet/förslaget för att undersöka någon ingenjörsaspekt och vars resultat kan ge ny eller kompletterande erfarenhet till ingenjörssamfundet och vetenskapen.\\slöser man en del eller hela delen av det ursprungliga problemet.\\Erfarenheten kommer ur en frågeställning som man i examensarbetet försöker besvara med tidigare och andras erfarenhet, egna eller modifierade metoder som ger ett resultat vilket kan användas för att diskutera ett svar på undersökningsfrågan.\\Detta stycke skall alltså, förutom det ursprungliga ”smala” problemet, innehålla  vad som skall undersökas för att skapa ny ingenjörserfarenhet och/eller vetenskap.}

\engExpl{The first paragraph after a heading is not indented, all of the
  subsequent paragraphs have their first line indented.}
  
This chapter describes the specific problem that this thesis addresses, the context of the problem, the
goals of this thesis project, and outlines the structure of the thesis.\\

\generalExpl{Give a general introduction to the area. (Remember to use appropriate references in this and all other sections.)}

% One can use either biblatex or bibtex - set as the option for the document at the top of this file
\ifbiblatex
\engExpl{We use the \emph{biblatex} package to handle our references.  We
use the command \texttt{parencite} to get a reference in parenthesis, like
this \textbackslash parencite\{heisenberg2015\} resulting in \parencite{heisenberg2015}.  It is also possible to include the author as part of the sentence using \texttt{textcite}, like talking about the work of \textbackslash textcite\{einstein2016\} resulting in \textcite{einstein2016}.\\
This also means that you have to change the include files to include biblatex and change the way that the reference.bib file is included.}
\else
\engExpl{We use the \emph{bibtex} package to handle our references.  We therefore
use the command \textbackslash cite\{farshin\_make\_2019\}. For example, Farshin, \etal described how to improve LLC
cache performance in \cite{farshin_make_2019} in the context of links running
at \qty{200}{Gbps}.}
\fi

\engExpl{Use the glossaries package to help yourself and your readers.
Add the acronyms and abbreviations to lib/acronyms.tex. Some examples are shown below:}
In this thesis we will examine the use of \glspl{LAN}. In this thesis we will
assume that \glspl{LAN} include \glspl{WLAN}, such as \gls{WiFi}.


\section{Background}
\label{sec:background}
\sweExpl{svensk: Bakgrund}

\generalExpl{Present the background for the area. Set the context for your project – so that your reader can understand both your project and this thesis. (Give detailed background information in Chapter 2 - together with related work.)
Sometimes it is useful to insert a system diagram here so that the reader
knows what are the different elements and their relationship to each
other. This also introduces the names/terms/… that you are going to use
throughout your thesis (be consistent). This figure will also help you later
delimit what you are going to do and what others have done or will do.}

As one can find in RFC 1235\,\cite{ioannidis_coherent_1991} multicast is useful for xxxx. A number of different \glspl{OS} have been used in this work, such as the following \glspl{OS}: UNIX, Linux, Windows, etc. The main focus will be on one \gls{OS}, namely Linux.

\section{Problem}
\label{sec:problem}
\sweExpl{svensk: Problemdefinition eller Frågeställning\\
Lyft fram det ursprungliga problemet om det finns något och definiera därefter
den ingenjörsmässiga erfarenheten eller/och vetenskapen som kan komma ur
projektet. }

Longer problem statement\\
If possible, end this section with a question as a problem statement.

% Research Question
\subsection{Original problem and definition}
\label{sec:researchQuestion}
\sweExpl{Ursprungligt problem och definition}
Some text

\subsection{Scientific and engineering issues}\sweExpl{Vetenskaplig och ingenjörsmässig frågeställning}
some text

\section{Purpose}
\sweExpl{Syfte}
\sweExpl{Skilj på syfte och mål! Syfte är att förändra något till det bättre. I examensarbetet finns ofta två aspekter på detta. Dels vill problemägaren (företaget) få sitt problem löst till det bättre men akademin och ingenjörssamfundet vill också få nya erfarenheter och vetskap. Beskriv ett syfte som tillfredställer båda dessa aspekter.\\
Det finns även ett syfte till som kan vara värt att beakta och det är att du som student skall ta examen och att du måste bevisa, i ditt examensarbete, att du uppfyller examensmålen. Dessa mål sammanfaller med kursmålen för examensarbetskursen. 
}
\generalExpl{State the purpose  of your thesis and the purpose of your degree project.\\
Describe who benefits and how they benefit if you achieve your goals. Include anticipated ethical, sustainability, social issues, etc. related to your project. (Return to these in your reflections in Section~\ref{sec:reflections}.)}



\section{Goals}
\sweExpl{Mål}
\sweExpl{Skilj på syfte och mål. Syftet är att åstakomma en förändring i något. Målen är vad som konkret skall göras för att om möjligt uppnå den önskade förändringen (syfte). }

\generalExpl{State the goal/goals of this degree project.}

The goal of this project is XXX. This has been divided into the following three sub-goals:
\begin{enumerate}
\item Subgoal 1 \sweExpl{för att tillfredsställa problemägaren – industrin?}
\item Subgoal 2\sweExpl{för att tillfredsställa ingenjörssamfundet och vetenskapen – akademin) }
\item Subgoal 3\sweExpl{eventuellt, för att uppfylla kursmålen – du som student}
\end{enumerate}

\generalExpl{In addition to presenting the goal(s), you might also state what the deliverables and results of the project are.}

\section{Research Methodology}\sweExpl{Undersökningsmetod}
\sweExpl{Här anger du vilken vilken övergripande undersökningsstrategi eller metod du skall använda för att försöka besvara den akademiska frågeställning och samtidigt lösa det e v ursprungliga problemet. Ofta kan man använda ”lösandet av ursprungsproblemet” som en fallstudie kring en akademisk frågeställning. Du undersöker någon intressant fråga i ”skarpt” läge och samlar resultat och erfarenhet ur detta.\\
Tänk på att företaget ibland måste stå tillbaka i sin önskan och förväntan på projektets resultat till förmån för ny eller kompletterande ingenjörserfarenhet och vetenskap (ditt examensarbete). Det är du som student som bestämmer och löser fördelningen mellan dessa två intressen men se till att alla är informerade. }
\generalExpl{Introduce your choice of methodology/methodologies and method/methods – and the reason why you chose them. Contrast them with and explain why you did not choose other methodologies or methods. (The details of the actual methodology and method you have chosen will be given in Chapter~\ref{ch:methods}. Note that in Chapter~\ref{ch:methods}, the focus could be research strategies, data collection, data analysis, and quality assurance.)\\
In this section you should present your philosophical assumption(s), research method(s), and research approach(es).}

\section{Delimitations}\sweExpl{Avgränsningar}
\generalExpl{Describe the boundary/limits of your thesis project and what you are explicitly not going to do. This will help you bound your efforts – as you have clearly defined what is out of the scope of this thesis project. Explain the delimitations. These are all the things that could affect the study if they were examined and included in the degree project.}

\section{Structure of the thesis}\sweExpl{ Rapportens disposition}
Chapter~\ref{ch:background} presents relevant background information about xxx.  Chapter~\ref{ch:methods} presents the methodology and method used to solve the problem. …



% ====================================================================================================
% ====================================================================================================


\cleardoublepage
\chapter{Background}
\label{ch:background}
\sweExpl{Bakgrund}

\generalExpl{When you do your literature study, you should have a nearly complete Chapters 1 and 2.\\
You may also find it convenient to introduce the future work section into your report early – so that you can put things that you think about but decide not to do now into this section.\\
Note that later you can move things between this future work section and what you have done as you may change your mind about what to do now versus what to put off to future work.
}
\generalExpl{What does a reader (another x student -- where x is your study line) need to know to understand your report?
What have others already done? (This is the “related work”.) Explain what and
how prior work/prior research will be applied on or used in the degree
project/work (described in this thesis). Explain why and what is not used in
the degree project and give valid reasons for rejecting the work/research.}

This chapter provides basic background information about xxx. Additionally, this chapter describes xxx. The chapter also describes related work xxxx.



\sweExpl{Vilken viktig litteratur och
  (forsknings-)artiklar har du studerat inom området (litteraturstudie)? }

\section{Major background area 1}
\sweExpl{Viktigt bakgrundsområde 1}
There are xxx characteristics that distinguish yyy from other information and communication technology (ICT) system, as shown in Figure~\ref{fig:lotsofstars}. Table \ref{tab:tablecaracteristics} summarizes these characteristics.

 
\begin{figure}[!ht]
  \begin{center}
    \includegraphics[width=0.5\textwidth]{figures/lots_of_stars.png}
  \end{center}
  \caption{Lots of stars  (Inspired by Figure x.y on page z of [xxx])}
  \label{fig:lotsofstars}
\end{figure}
\sweExpl{Massor av stärnor (Inspirerad av figur x.y på sidan z i [xxx])}


\begin{table}[!ht]
  \begin{center}
    \caption{xxx characteristics}
    \label{tab:tablecaracteristics}
    \begin{tabular}{l|S[table-format=4.6]} % <-- Alignments: 1st column left, 2nd middle, with vertical lines in between
      \textbf{Characteristics} & \textbf{Description}\\
      $\alpha$ & $\beta$ \\
      \hline
      1 & 1110.1\\
      2 & 10.1\\
      3 & 23.113231\\
    \end{tabular}
  \end{center}
\end{table}
\sweExpl{Egenskaper}
\sweExpl{Beskrivning}

\subsection{Subarea 1.1}
Entangled states are an important part of quantum cryptography, but also relevant in other domains. This concept might be relevant for neutrinos, see for example \cite{kim_small-mass_2016}.

\subsection{Subarea 1.1.2}
Computational methods are increasingly used as a third method of carrying out
scientific investigations. For example, computational experiments were used to
find the amount of wear in a polyethylene liner of a hip prosthesis in \cite{maguire_jr_new_2014}.
…

\subsection{Subarea 1.1.2}
Using the nearest data center may improve performance, see \cite{bogdanov_nearest_2015}


\subsection{Link layer Encapsulation}
\label{sec:llencap}

See Figure~\ref{fig:ieee8023-data-packet} which uses the \textsf{bytefield}  \LaTeX\ package. 


\begin{figure}[!ht]
	\centering
\begin{bytefield}{21}
\bitbox[]{7}{} & \bitbox[]{3}{\tiny octets:} & \bitbox[]{4}{\tiny 6} & \bitbox[]{4}{\tiny 6} & \bitbox[]{3}{\tiny 2} & \bitbox[]{5}{\tiny 46 to 1500} & \bitbox[]{3}{\tiny 0 to 46} & \bitbox[]{2}{\tiny 4}\\ 

\bitbox[]{8}{\textbf{ETHERNET \\[-1ex] \tiny{data link-layer}}} & \bitbox[]{2}{} & 

\bitbox{4}{\tiny Destination Address} & \bitbox{4}{\tiny Source Address} & \bitbox{3}{\tiny Length/ Type} & 
\bitbox{5}{\tiny Data Payload} & \bitbox{3}{\tiny Padding} &
\bitbox{2}{\tiny CRC} \\

\bitbox[]{1}{} &\bitbox[]{3}{\tiny octets:} & \bitbox[]{4}{\tiny 7} & \bitbox[]{2}{\tiny 1} & \bitbox[]{0}{$\vdots$ \\[1ex]} & \bitbox[]{16}{} & \bitbox[]{0}{$\vdots$ \\[1ex]} & \bitbox[]{5}{} & \bitbox[]{4}{\tiny Variable}\\

\bitbox[]{4}{\textbf{MAC \\[-1ex] \tiny{packet}}} & \colorbitbox{lightgray}{4}{\tiny Preamble} & \colorbitbox{lightgray}{2}{\tiny SFD} & \colorbitbox{lightgray}{16}{\tiny MAC Client Data} & \colorbitbox{lightgray}{3}{\tiny Padding} &
\colorbitbox{lightgray}{2}{\tiny CRC} & \colorbitbox{lightgray}{4}{\tiny Extension}
\end{bytefield}
     \caption{Ethernet data link layer protocol encapsulated into a IEEE~802.3 MAC packet}
     \label{fig:ieee8023-data-packet}
\end{figure}

\subsection{IP packet headers}
\label{sec:ipheaders}
The data link layer will receive a packet from the IP layer. The layout of
an IPv4 packet is shown in Figure~\ref{fig:ipv4-header}. This should be
contrasted with the IPv6 header shown in Figure~\ref{fig:ipv6-header}.

%
% IPv4 packet header
%
\begin{figure}[!ht]
	\centering
\begin{bytefield}{32}
\bitheader{0-31} \\
\bitbox{4}{\footnotesize{Version}} & \bitbox{4}{IHL} & \bitbox{6}{\tiny{Type of Service}} & \bitbox{2}{{\scriptsize ECN}} \bitbox{16}{Total Length}\\
\bitbox{16}{Identification} & \bitbox{3}{Flags} & \bitbox{13}{Fragment Offset}\\
\bitbox{8}{Time to Live} & \bitbox{8}{Protocol} & \bitbox{16}{Header Checksum}\\
\wordbox{1}{Source Address}\\
\wordbox{1}{Destination Address}\\
\colorbitbox{lightgray}{24}{Options} & \colorbitbox{lightgray}{8}{Padding}
\end{bytefield}
     \caption[IPv4 datagram header]{IPv4 datagram header. Light grey coloured fields are optional.}
    \label{fig:ipv4-header} 
\end{figure}

%
% IPv6 packet header
%
\begin{figure}[!ht]
	\centering
\begin{bytefield}{32}
\bitheader{0-31} \\
\bitbox{4}{\footnotesize{Version}} & \bitbox{8}{Traffic Class} & \bitbox{20}{Flow Label}\\
\bitbox{16}{Payload Length} & \bitbox{8}{Next Header} & \bitbox{8}{Hop Limit}\\
\wordbox{4}{Source Address}\\
\wordbox{4}{Destination Address}\\
\end{bytefield}
     \caption{IPv6 datagram header}
    \label{fig:ipv6-header}
\end{figure}

\subsection{Test for accessibility of formulas}

As can be seen in these equations:
$c=2 \cdot \pi \cdot r$ or \[ \int_{a}^{b} x^2 \,dx \] a chemical formula: $(C_5O_2H_8)_n$
...
\section{Major background area 2}\sweExpl{Viktigt bakgrundsområde 2}
...
\subsection{\glsentryshort{WLAN} Security}% you can't use the \gls() command in a heading - but you can get the short (\glsentryshort) or long version (\glsentryshort) or \glsentrylong or even the text entry (\glsentrytext) and then there is no problem - see https://tex.stackexchange.com/questions/198140/glossaries-and-custom-section-headings-broken

\subsection{Network layer security}
...

\section{Related work area}\sweExpl{Relaterade arbeten}


\subsection{Major related work 1}\sweExpl{Relaterade arbeten 1}
Carrier clouds have been suggested as a way to reduce the delay between the users and the cloud server that is providing them with content. However, there is a question of how to find the available resources in such a carrier cloud. One approach has been to disseminate resource information using an extension to OSPF-TE, see Roozbeh, Sefidcon, and Maguire \cite{roozbeh_resource_2013}.


\subsection{Major related work n}\sweExpl{Relaterade arbeten}

\subsection{Minor related work 1}\sweExpl{Mindre relaterat arbete 1}


…
\subsection{Minor related work n}\sweExpl{Mindre relaterat arbete n}


\section{Summary}\sweExpl{Sammanfattning}
\sweExpl{Det är trevligt om detta kapitel
  avslutas med en sammanfattning. Till exempel kan du inkludera en tabell som
  sammanfattar andras idéer och fördelar och nackdelar med varje - så som
  senare kan du jämföra din lösning till var och en av dessa. Detta kommer
  också att hjälpa dig att definiera de variabler som du kommer att använda
  för din utvärdering.}

\engExpl{It is nice to have this chapter conclude with a summary. For
  example, you can include a table that summarizes other people's ideas and
  benefits and drawbacks with each - so as later you can compare your solution
  to each of them. This will also help you define the variables that you will
  use for your evaluation.}



% ====================================================================================================
% ====================================================================================================


\cleardoublepage
\chapter{Method or Methods}
\label{ch:methods}
\sweExpl{Metod eller Metodval}
\generalExpl{This chapter is about Engineering-related
  content, Methodologies and Methods.  Use a self-explaining title.\\The
  contents and structure of this chapter will change with your choice of
  methodology and methods.}



\generalExpl{Describe the engineering-related contents (preferably with models) and the research methodology and methods that are used in the degree project.\\
Give a theoretical description of the scientific or engineering methodology  you are going to use and why have you chosen this method. What other methods did you consider and why did you reject them.\\
In this chapter, you describe what engineering-related and scientific skills you are going to apply, such as modeling, analyzing, developing, and evaluating engineering-related and scientific content. The choice of these methods should be appropriate for the problem . Additionally, you should be consciousness of aspects relating to society and ethics (if applicable). The choices should also reflect your goals and what you (or someone else) should be able to do as a result of your solution - which could not be done well before you started.}

The purpose of this chapter is to provide an overview of the research method
used in this thesis. Section~\ref{sec:researchProcess} describes the research
process. Section~\ref{sec:researchParadigm} details the research
paradigm. Section~\ref{sec:dataCollection} focuses on the data collection
techniques used for this research. Section~\ref{sec:experimentalDesign}
describes the experimental design. Section~\ref{sec:assessingReliability}
explains the techniques used to evaluate the reliability and validity of the
data collected. Section~\ref{sec:plannedDataAnalysis} describes the method
used for the data analysis. Finally, Section~\ref{sec:evaluationFramework}
describes the framework selected to evaluate xxx.

\sweExpl{Vilka vetenskaplig eller ingenjörs-metodik ska du använda och varför har du valt den här metoden. Vilka andra metoder gjorde du övervägde du och varför du avvisar dem.
Vad är dina mål? (Vad ska du kunna göra som ett resultat av din lösning - vilken inte kan göras i god tid innan du började)
Vad du ska göra? Hur? Varför? Till exempel, om du har implementerat en artefakt vad gjorde du och varför? Hur kommer du utvärdera den.
Syftet med detta kapitel är att ge en översikt över forsknings metod som
används i denna avhandling. Avsnitt~\ref{sec:researchProcess} beskriver forskningsprocessen. Avsnitt~\ref{sec:researchParadigm} beskriver forskningsparadigmen detaljerat. Avsnitt~\ref{sec:dataCollection} fokuserar på datainsamlingstekniker som används för denna forskning. Avsnitt~\ref{sec:experimentalDesign} beskriver experimentell
design. Avsnitt~\ref{sec:assessingReliability} förklarar de tekniker som används för att utvärdera
tillförlitligheten och giltigheten av de insamlade uppgifterna. Avsnitt~\ref{sec:plannedDataAnalysis}
beskriver den metod som används för dataanalysen. Slutligen, Avsnitt~\ref{sec:evaluationFramework}
beskriver ramverket som valts för att utvärdera xxx.\\
Ofta kan man koppla ett antal följdfrågor till undersökningsfrågan och problemlösningen t ex\\
(1) Vilken process skall användas för konstruktion av lösningen och vilken process skall kopplas till denna för att svara på undersökningsfrågan?\\
(2) Hur och vilket resultat (storheter) skall presenteras både för att redovisa svar på undersökningsfrågan (resultatkapitlet i denna rapport) och redovisa resultat av problemlösningen (prototypen, ofta dokument som bilagor men vilka dokument och varför?).\\
(3) Vilken teori/teknik skall väljas och användas både för undersökningen (taxonomi, matematik, grafer, storheter mm)  och  problemlösning (UML, UseCases, Java mm) och varför?\\
(4) Vad behöver du som student leverera för att uppnå hög kvaliet (minimikrav) eller mycket hög kvalitet på examensarbetet?\\
(5) Frågorna kopplar till de följande underkapitlen.\\
(6) Resonemanget bygger på att studenter på hing-programmet ofta skall konstruera något åt problemägaren och att man till detta måste koppla en intressant ingenjörsfråga. Det finns hela tiden en dualism mellan dessa aspekter i exjobbet.
}

\section{Research Process}
\label{sec:researchProcess}

\sweExpl{Undersökningsrocess och utvecklingsprocess}

Figure~\ref{fig:researchprocess} shows the steps conducted in order to carry out this research. 

\sweExpl{Figur~\ref{fig:researchprocess} visar de steg som utförs för att genomföra\\
Beskriv, gärna med ett aktivitetsdiagram (UML?), din undersökningsprocess och utvecklingsprocess.  Du måste koppla ihop det akademiska intresset (undersökningsprocess) med ursprungsproblemet (utvecklingsprocess)
denna forskning.\\
Aktivitetsdiagram från t ex UML-standard}


 
\begin{figure}[!ht]
  \begin{center}
    \includegraphics[width=0.5\textwidth]{figures/researchprocess.png}
  \end{center}
  \caption{Research Process}
  \label{fig:researchprocess}
\end{figure}

\generalExpl{Example of using customize item labels.}
Some steps in the process:
\begin{enumerate}[leftmargin=*, label=\textbf{Step \arabic*}, ref=Step \arabic*] %labelindent=1em for indent
    \itemsep0em
    \item \label{x:s1} plan experiment,
    \item \label{x:s2} conduct experiment,
    \item \label{x:s3} analyze data from the experiment, and
    \item \label{x:s4} discuss the results of the analysis.
\end{enumerate}

\sweExpl{Forskningsprocessen}


\section{Research Paradigm}
\label{sec:researchParadigm}
\sweExpl{Undersökningsparadigm\\
Exempelvis\\
Positivistisk (vad/hur fungerar det?) kvalitativ fallstudie med en deduktivt (förbestämd) vald ansats och ett induktivt(efterhand uppstår dataområden och data) insamlade av data och erfarenheter.}


\section{Data Collection}
\label{sec:dataCollection}
\sweExpl{Datainsamling\\
(Detta bör också visa att du är medveten om de sociala och etiska frågor som
kan vara relevanta för dina data insamlingsmetod.)}
\generalExpl{This should also show that you are aware of the social and ethical concerns that might be relevant to your data collection method.}



\subsection{Sampling}
\sweExpl{Stickprovsundersökning}

\subsection{Sample Size}
\sweExpl{Provstorleken}

\subsection{Target Population}
\sweExpl{Målgruppen}

\section[Experimental design/Planned Measurements]{Experimental design and\\Planned Measurements}
\label{sec:experimentalDesign}
\sweExpl{Experimentdesign/Mätuppställning}

\subsection{Test environment/test bed/model}
\engExpl{Describe everything that someone else would need to reproduce your test environment/test bed/model/… .}
\sweExpl{Testmiljö/testbädd/modell\\
Beskriv allt att någon annan skulle behöva återskapa din testmiljö / testbädd / modell / …}

\subsection{Hardware/Software to be used}
\sweExpl{Hårdvara / programvara som ska användas}


\section{Assessing reliability and validity of the data collected}
\label{sec:assessingReliability}
\sweExpl{Bedömning av validitet och reliabilitet hos använda metoder och insamlade data }


\subsection{Validity of method}
\label{sec:validtyOfMethod}
\sweExpl{Giltigheten av metoder\\
  Har dina metoder gett dig de rätta svaren och lösningarna? Var metoderna korrekta?}

\engExpl{How will you know if your results are valid?}
\engExpl{Remember that validity is about the \textit{accuracy} of a measurement while reliability is about the \textit{consistency} of the measurement values under the same conditions (\ie repeatability).}

\subsection{Reliability of method}
\label{sec:reliabilityOfMethod}
\sweExpl{Tillförlitlighet av för metoder\\
Hur bra är dina metoder, finns det bättre metoder? Hur kan du förbättra dem?}
\engExpl{How will you know if your results are reliable?}

\subsection{Data validity}
\label{sec:dataValidity}
\sweExpl{Giltigheten av uppgifter\\
Hur vet du om dina resultat är giltiga? Är ditt resultat rättvisande?}

\subsection{Reliability of data}
\label{sec:reliabilityOfData}
\sweExpl{Tillförlitlighet av data\\
Hur vet du om dina resultat är tillförlitliga? Hur bra är dina resultat?}


\section{Planned Data Analysis}
\label{sec:plannedDataAnalysis}
\sweExpl{Metod för analys av data}


\subsection{Data Analysis Technique}
\label{sec:dataAnalysisTechnique}
\sweExpl{Dataanalysteknik}

\subsection{Software Tools}
\label{sec:softwareTools}
\sweExpl{Mjukvaruverktyg}


\section{Evaluation framework}
\label{sec:evaluationFramework}
\sweExpl{Utvärdering och ramverk\\
Metod för utvärdering, jämförelse mm. Kopplar till kapitel~\ref{ch:resultsAndAnalysis}.}

\section{System documentation}
\label{sec:systemDocumentation}
\sweExpl{Systemdokumentation\\
Med vilka dokument och hur skall en konstruerad prototyp dokumenteras? Detta blir ofta bilagor till rapporten och det som problemägaren till det ursprungliga problemet (industrin) ofta vill ha.\\
Bland dessa bilagor återfinns ofta, och enligt någon angiven standard, kravdokument, arkitekturdokument, designdokumnet, implementationsdokument, driftsdokument, testprotokoll mm.}
\generalExpl{If this is going to be a complete document consider putting it in as an appendix, then just put the highlights here.}




% ====================================================================================================
% ====================================================================================================


\cleardoublepage
\chapter{What you did}\engExpl{Choose your own chapter title to describe this}
\label{ch:whatYouDid}
\sweExpl{[Vad gjorde du? Hur gick det till? – Välj lämplig rubrik (“Genomförande”, “Konstruktion”, ”Utveckling”  eller annat]}


\engExpl{What have you done? How did you do it? What design decisions did you make? How did what you did help you to meet your goals?}
\sweExpl{Vad du har gjort? Hur gjorde du det? Vilka designval gjorde du?\\
Hur kom det du hjälpte dig att uppnå dina mål?}

% the following sets the TOC entry to break after the & - note you have to include the first letter of the following word as it get swolled by the \texorpdfstring{}{} processing
\section[Hardware/Software design …/Model/Simulation model \&\texorpdfstring{\\}{ p} parameters/…]{Hardware/Software design …/Model/Simulation model \& parameters/…}
\sweExpl{Hårdvara / Mjukvarudesign ... / modell / Simuleringsmodell och parametrar / …}

Figure~\ref{fig:homepageicon} shows a simple icon for a home page. The time
to access this page when served will be quantified in a series of
experiments. The configurations that have been tested in the test bed are
listed in Table~\ref{tab:configstested}. In \SI{7.0}{\percent} of cases there was an error indicating xxxxx.

\sweExpl{Figur~\ref{fig:homepageicon}  visar en enkel ikon för en hemsida. Tiden för att få tillgång till den här sidan när den laddas kommer att kvantifieras i en serie experiment. De konfigurationer som har testats i provbänk listas ini tabell~\ref{tab:configstested}.\\
Vad du har gjort? Hur gjorde du det? Vilka designval gjorde du?}
 
\begin{figure}[!ht]
  \begin{center}
    \includegraphics[width=0.25\textwidth]{figures/Homepage-icon.png}
  \end{center}
  \caption{Homepage icon}
  \label{fig:homepageicon}
\end{figure}

\begin{table}[!ht]
  \begin{center}
    \caption{Configurations tested}
    \label{tab:configstested}
    \resizebox{\columnwidth}{!}{%
    \begin{tabular}{l|c} % <-- Alignments: 1st column left, 2nd middle and 3rd right, with vertical lines in between
      \textbf{Configuration} & \textbf{Description} \\
      \hline
      1 & Simple test with one server\\
      2 & Simple test with one server\\
    \end{tabular}
    }
  \end{center}
\end{table}
\sweExpl{Testade konfigurationer}

\section{Implementation …/Modeling/Simulation/…}
\label{sec:implementationDetails}
\sweExpl{Implementering … / modellering / simulering / …}

Two commonly used simulators are:
\begin{description}[labelwidth =\widthof{\textbf{ns-2 or ns-3 simulator}}, leftmargin = !]
    \item[\textbf{Mininet}] This simulator uses traffic control (\texttt{tc}) to simulate network devices connected by links with specific bandwidth, packet loss rates, qdisc methods, etc.
    
    
    \item[\textbf{ns-2 or ns-3 simulator}] These simulators are very useful for simulating wireless communication links between moving devices. You can specific the mobility patterns of the nodes.
\end{description}

\subsection{Some examples of coding}
\engExpl{This section is simply to show some example of how you can include code in your thesis - this is not a section you would have in your thesis.}
\sweExpl{Det här avsnittet är helt enkelt för att visa ett exempel på hur du kan inkludera kod i ditt examensarbete - det här är inte ett avsnitt du skulle ha i ditt examensarbete.}

Listing~\ref{lst:helloWorldInC} shows an example of a simple program written
in C code.

\begin{lstlisting}[language={C}, caption={Hello world in C code}, label=lst:helloWorldInC]
int main() {
printf("hello, world");
return 0;
}
\end{lstlisting}


In contrast, Listing~\ref{lst:programmes} is an example of code in Python to
get a list of all of the programs at KTH.

\lstset{extendedchars=true}  %% This allows characters codes in the range 128-255
\begin{lstlisting}[language={Python}, caption={Using a python program to
    access the KTH API to get all of the programs at KTH}, label=lst:programmes]
KOPPSbaseUrl = 'https://www.kth.se'

def v1_get_programmes():
    global Verbose_Flag
    #
    # Use the KOPPS API to get the data
    # note that this returns XML
    url = "{0}/api/kopps/v1/programme".format(KOPPSbaseUrl)
    if Verbose_Flag:
        print("url: " + url)
    #
    r = requests.get(url)
    if Verbose_Flag:
        print("result of getting v1 programme: {}".format(r.text))
    #
    if r.status_code == requests.codes.ok:
        return r.text           # simply return the XML
    #
    return None
\end{lstlisting}
\FloatBarrier

\subsection{Some examples of figures in tikz}
\engExpl{This section is simply to show some example of how you can draw your own figures for in your thesis - this is not a section you would have in your thesis.}
\sweExpl{Det här avsnittet är helt enkelt för att visa ett exempel på hur du kan rita dina egna figurer i ditt examensarbete – det här är inte ett avsnitt du skulle ha i ditt examensarbete.}

These figures are just some examples to show that you can draw your own figures for in your thesis. This has two advantages: \first you do not have to worry about copyrights -- as these are your own figures and \Second the text is now readable and not simply a picture of text -- so screen readers can read the figure's contents to someone who is listening to the contents of your thesis.

\subsubsection{Azure's Form Recognizer}
\Cref{fig:processAnInvoice} shows the processing of key-value extraction from a PDF document using Azure's Form Recognizer. 

\tikzstyle{processBox} = [rectangle, rounded corners, minimum width=3cm, minimum height=1cm,text centered, draw=black, fill=red!20]
\tikzstyle{largeBox} = [rectangle, rounded corners, minimum width=3cm, minimum height=4cm,text centered, draw=black]
\begin{figure}[!ht]
\resizebox{1.1\textwidth}{!}{%
\begin{tikzpicture}
[align=left,node distance=2cm]

\node (document) [tape,tape bend top=none,draw,font=\sffamily] {PDF\\Document};
\node (GDM) [processBox,  right=0.5cm of document] {OCR};
\node (OCRoutput) [largeBox, right=1cm of GDM] {OCR output};

\node (kvp) [tape,tape bend top=none,draw,font=\sffamily, below=0.25cm of OCRoutput.north] {key-value\\pairs};
\node (entities) [tape,tape bend top=none,draw,font=\sffamily, above=0.35cm of OCRoutput.south] {Entities};
\node (Manual) [processBox, right=1cm of kvp] {Analyze the extracted\\key-value pairs};
\draw [-latex](document) --  (GDM);
\draw [-latex](kvp) --  (Manual);
\path[ draw
     , -latex'] let \p1=(GDM.east), \p2=(kvp.west) in (GDM.east) -- +(0.25*\x2-0.25*\x1, \y1) -- +(0.5*\x2-0.5*\x1, \y2) -- (kvp.west);
\path[ draw
     , -latex'] let \p1=(GDM.east), \p2=(kvp.west), \p3=(entities.west) in (GDM.east) --  +(0.25*\x2-0.25*\x1, \y1) -- +(0.5*\x3-0.5*\x1, \y3) -- (entities.west);
\end{tikzpicture}
}
\caption{The processing of key-value extraction from a PDF document using Azure's Form Recognizer}
  \label{fig:processAnInvoice}
\end{figure}
\FloatBarrier
\subsubsection{Hyper-V with Containers}
 \Cref{fig:hyperVcontainers} shows how Hyper-V deals with containers.
 
\tikzstyle{container} = [rectangle, rounded corners, minimum width=2cm, minimum height=1cm,text centered, draw=black, fill=blue!20]
\tikzstyle{containerization} = [rectangle, rounded corners, minimum width=13.25cm, minimum height=1cm,text centered, draw=black, fill=blue!20]
\tikzstyle{hypervisor} = [rectangle, rounded corners, minimum width=13.25cm, minimum height=1cm,text centered, draw=black, fill=red!20]
\tikzstyle{os} = [rectangle, rounded corners, minimum width=13.25cm, minimum height=1cm,text centered, draw=black, fill=orange!20]
\tikzstyle{guestos} = [rectangle, rounded corners, minimum width=2cm, minimum height=1cm,text centered, draw=black, fill=orange!40]
\tikzstyle{infrastructure} = [rectangle, rounded corners, minimum width=13.25cm, minimum height=1cm,text centered, draw=black, fill=green!20]

\tikzstyle{hos} = [rectangle, rounded corners, minimum width=6cm, minimum height=1cm,text centered, draw=black, fill=orange!20]
\tikzstyle{kernel} = [rectangle, rounded corners, minimum width=6cm, minimum height=1cm,text centered, draw=black, fill=purple!20]
\tikzstyle{services} = [rectangle, rounded corners, minimum width=3cm, minimum height=1cm,text centered, draw=black, fill=pink!20]
\begin{figure}[ht!]
    \centering
\resizebox{1\textwidth}{!}{%
\begin{tikzpicture}
[align=center,node distance=2cm]

\node (Infrastructure) [infrastructure, text width=13cm, text centered] {Infrastructure};
\node (OS1) [hos, anchor=north west, align=left, above=1.5cm of Infrastructure.north west, anchor=north west, text width=6cm, text centered] {Host OS};

\node (OS2) [hos, anchor= west, align=left, right=0.5cm of OS1.east, text width=6cm, anchor= west, text centered] {Host OS};

\node (Kernel1) [kernel, anchor=north west, align=left, above=1.5cm of OS1.north east, anchor=north east, text width=3cm, text centered] {Kernel};

\node (Kernel2) [kernel, anchor=north west, align=left, above=1.5cm of OS2.north east, anchor=north east, text width=3cm, text centered] {Kernel};

\node (ServiceA) [container, anchor=east, above=1 cm of Kernel1.east, anchor=east] {Services};
\node (AppA) [container,  left=0.25cm of ServiceA] {App 1};

\node (ServiceB) [container, anchor=east, above=1 cm of Kernel2.east, anchor=east] {Services};
\node (AppB) [container,  left=0.25cm of ServiceB] {App 2};
%\node (AppC) [container,  right=0.25cm of AppB] {App 3};

\draw[black,thick,dashed] ($(OS2.north west)+(-0.3,3.75)$)  rectangle ($(OS2.south east)+(0.5,-0.3)$);
\node[text width=5cm, text=red, above=0.1cm of ServiceB] 
    {\textbf{Container}};

\draw[red,thick,dotted] ($(Kernel2.north west)+(-0.3,1.6)$)  rectangle ($(Kernel2.south east)+(0.3,-0.3)$);
\node[text width=5cm, text=black, above=0.8cm of ServiceB] 
    {\textbf{VM}};
\end{tikzpicture}
}
    \caption{Hyper-V with containers}
    \label{fig:hyperVcontainers}
\end{figure}
\FloatBarrier
\subsubsection{\glsfmtshort{VM} versus Containers}
\Cref{fg:vmsVersusContainers} shows a comparison of virtual machines (VMs) versus containers.

\begin{figure*}[ht!]
    \centering
    \begin{subfigure}[t]{0.5\textwidth}
        \centering
\resizebox{1\textwidth}{!}{%
\begin{tikzpicture}
[align=left,node distance=2cm]

\node (AppA) [container,align=left] {App 1};
\node (AppB) [container,  right=0.25cm of AppA] {App 2};
\node (AppC) [container,  right=0.25cm of AppB] {App 3};

\node (GosA) [guestos,align=left,  below=0.25cm of AppA.south west,anchor=north west] {Guest OS};
\node (GosB) [guestos,  right=0.25cm of GosA] {Guest OS};
\node (GosC) [guestos,  right=0.25cm of GosB] {Guest OS};

\draw [decoration={brace,amplitude=0.5em},decorate, ultra thick,gray, transform canvas={xshift = 0.5cm}]
       (AppC.north -| AppC.east) -- (GosC.south -| AppC.east);
\node[text width=5cm,  right=1cm of GosC.north east] 
    {\textbf{VMs}};

\node (Hypervisor) [hypervisor, anchor=north west, align=left, below=0.25cm of GosA.south west, anchor=north west, text width=13cm, text centered] {Hypervisor};

\node (OS) [os, anchor=north west, align=left, below=0.25cm of Hypervisor.south west, anchor=north west, text width=13cm, text centered] {Host OS};

\node (Infrastructure) [infrastructure, anchor=north west, align=left, below=0.25cm of OS.south west, anchor=north west, text width=13cm, text centered] {Infrastructure};


\end{tikzpicture}
}
        \caption{VM}
    \end{subfigure}%
    ~ 
    \begin{subfigure}[t]{0.5\textwidth}
        \centering
        \resizebox{1\textwidth}{!}{%
\begin{tikzpicture}
[align=left,node distance=2cm]

\node (AppA) [container,align=left] {App 1};
\node (AppB) [container,  right=0.25cm of AppA] {App 2};
\node (AppC) [container,  right=0.25cm of AppB] {App 3};
\node[text width=5cm,  right=0.25cm of AppC] 
    {\textbf{Apps running in Containers}};


\node (Containerization) [containerization, anchor=north west, align=left, below=0.25cm of AppA.south west, anchor=north west, text width=13cm, text centered] {Docker Engine};

\node (OS) [os, anchor=north west, align=left, below=0.25cm of Containerization.south west, anchor=north west, text width=13cm, text centered] {Host OS};

\node (Infrastructure) [infrastructure, anchor=north west, align=left, below=0.25cm of OS.south west, anchor=north west, text width=13cm, text centered] {Infrastructure};


\end{tikzpicture}
}
        \caption{Containers}
    \end{subfigure}
    \caption{Virtual machines (VMs) versus Containers}
    \label{fg:vmsVersusContainers}
\end{figure*}



% ====================================================================================================
% ====================================================================================================


\cleardoublepage
\chapter{Results and Analysis}
\label{ch:resultsAndAnalysis}

\sweExpl{svensk: Resultat och Analys}

\engExpl{Sometimes this is split into two chapters.\\Keep in mind: How you are going to evaluate what you have done? What are your metrics?\\Analysis of your data and proposed solution\\Does this meet the goals which you had when you started?}

In this chapter, we present the results and discuss them.

\sweExpl{I detta kapitel presenterar vi resultaten och diskutera dem.\\Ibland delas detta upp i två kapitel.\\Hur du ska utvärdera vad du har gjort? Vad är din statistik?\\Analys av data och föreslagen lösning\\Innebär detta att uppfyllelse av de mål som du hade när du började?}

\section{Major results}
\sweExpl{Huvudsakliga resultat}

Some statistics of the delay measurements are shown in Table~\ref{tab:delayMeasurements}.
The delay has been computed from the time the GET request is received until the response is sent.

\sweExpl{Lite statistik av fördröjningsmätningarna visas i Tabell~\ref{tab:delayMeasurements}. Förseningen har beräknats från den tidpunkt då begäran GET tas emot fram till svaret skickas.}

\begin{table}[!ht]
  \begin{center}
    \caption{Delay measurement statistics}
    \label{tab:delayMeasurements}
    \begin{tabular}{l|S[table-format=4.2]|S[table-format=3.2]} % <-- Alignments: 1st column left, 2nd middle and 3rd right, with vertical lines in between
      \textbf{Configuration} & \textbf{Average delay (ns)} & \textbf{Median delay (ns)}\\
      \hline
      1 & 467.35 & 450.10\\
      2 & 1687.5 & 901.23\\
    \end{tabular}
  \end{center}
\end{table}

Table \ref{tab:ping_results} shows the measurement of round trip times from four hosts to and from a server.
\begin{table}[ht!]
\caption[RTT for 4 hosts]{Result for the ping measurements of RTT for 4 hosts} 
\label{tab:ping_results}
\vspace{1em}
\centering
\begin{tabular}{l *{4}{S[table-format=2.3]}}
{} & \multicolumn{4}{c}{host to server RTT in ms} \\
\cmidrule{2-5}
Host & \multicolumn{1}{c}{min}  & \multicolumn{1}{c}{avg} & \multicolumn{1}{c}{max} & \multicolumn{1}{c}{mdev} \\
\midrule
h1 & 5.625 & 5.625 & 5.625 & 0.0 \\
h2 & 2.909 & 2.909 & 1.909 & 0.0 \\
h3 & 5.007 & 5.007 & 5.007 & 0.0 \\
h4 & 2.308 & 2.308 & 2.308 & 0.0 \\
\midrule
\end{tabular}
\end{table}
\FloatBarrier

\sweExpl{Fördröj mätstatistik}
\sweExpl{Konfiguration | Genomsnittlig fördröjning (ns) | Median fördröjning (ns)}

Figure \ref{fig:processing_vs_payload_length} shows an example of the performance as measured in the experiments.

\begin{figure}[!ht]
% GNUPLOT: LaTeX picture
\setlength{\unitlength}{0.240900pt}
\ifx\plotpoint\undefined\newsavebox{\plotpoint}\fi
\begin{picture}(1500,900)(0,0)
\sbox{\plotpoint}{\rule[-0.200pt]{0.400pt}{0.400pt}}%
\put(171.0,131.0){\rule[-0.200pt]{4.818pt}{0.400pt}}
\put(151,131){\makebox(0,0)[r]{ 1.5}}
\put(1419.0,131.0){\rule[-0.200pt]{4.818pt}{0.400pt}}
\put(171.0,212.0){\rule[-0.200pt]{4.818pt}{0.400pt}}
\put(151,212){\makebox(0,0)[r]{ 2}}
\put(1419.0,212.0){\rule[-0.200pt]{4.818pt}{0.400pt}}
\put(171.0,292.0){\rule[-0.200pt]{4.818pt}{0.400pt}}
\put(151,292){\makebox(0,0)[r]{ 2.5}}
\put(1419.0,292.0){\rule[-0.200pt]{4.818pt}{0.400pt}}
\put(171.0,373.0){\rule[-0.200pt]{4.818pt}{0.400pt}}
\put(151,373){\makebox(0,0)[r]{ 3}}
\put(1419.0,373.0){\rule[-0.200pt]{4.818pt}{0.400pt}}
\put(171.0,454.0){\rule[-0.200pt]{4.818pt}{0.400pt}}
\put(151,454){\makebox(0,0)[r]{ 3.5}}
\put(1419.0,454.0){\rule[-0.200pt]{4.818pt}{0.400pt}}
\put(171.0,534.0){\rule[-0.200pt]{4.818pt}{0.400pt}}
\put(151,534){\makebox(0,0)[r]{ 4}}
\put(1419.0,534.0){\rule[-0.200pt]{4.818pt}{0.400pt}}
\put(171.0,615.0){\rule[-0.200pt]{4.818pt}{0.400pt}}
\put(151,615){\makebox(0,0)[r]{ 4.5}}
\put(1419.0,615.0){\rule[-0.200pt]{4.818pt}{0.400pt}}
\put(171.0,695.0){\rule[-0.200pt]{4.818pt}{0.400pt}}
\put(151,695){\makebox(0,0)[r]{ 5}}
\put(1419.0,695.0){\rule[-0.200pt]{4.818pt}{0.400pt}}
\put(171.0,776.0){\rule[-0.200pt]{4.818pt}{0.400pt}}
\put(151,776){\makebox(0,0)[r]{ 5.5}}
\put(1419.0,776.0){\rule[-0.200pt]{4.818pt}{0.400pt}}
\put(171.0,131.0){\rule[-0.200pt]{0.400pt}{4.818pt}}
\put(171,90){\makebox(0,0){ 0}}
\put(171.0,756.0){\rule[-0.200pt]{0.400pt}{4.818pt}}
\put(298.0,131.0){\rule[-0.200pt]{0.400pt}{4.818pt}}
\put(298,90){\makebox(0,0){ 10}}
\put(298.0,756.0){\rule[-0.200pt]{0.400pt}{4.818pt}}
\put(425.0,131.0){\rule[-0.200pt]{0.400pt}{4.818pt}}
\put(425,90){\makebox(0,0){ 20}}
\put(425.0,756.0){\rule[-0.200pt]{0.400pt}{4.818pt}}
\put(551.0,131.0){\rule[-0.200pt]{0.400pt}{4.818pt}}
\put(551,90){\makebox(0,0){ 30}}
\put(551.0,756.0){\rule[-0.200pt]{0.400pt}{4.818pt}}
\put(678.0,131.0){\rule[-0.200pt]{0.400pt}{4.818pt}}
\put(678,90){\makebox(0,0){ 40}}
\put(678.0,756.0){\rule[-0.200pt]{0.400pt}{4.818pt}}
\put(805.0,131.0){\rule[-0.200pt]{0.400pt}{4.818pt}}
\put(805,90){\makebox(0,0){ 50}}
\put(805.0,756.0){\rule[-0.200pt]{0.400pt}{4.818pt}}
\put(932.0,131.0){\rule[-0.200pt]{0.400pt}{4.818pt}}
\put(932,90){\makebox(0,0){ 60}}
\put(932.0,756.0){\rule[-0.200pt]{0.400pt}{4.818pt}}
\put(1059.0,131.0){\rule[-0.200pt]{0.400pt}{4.818pt}}
\put(1059,90){\makebox(0,0){ 70}}
\put(1059.0,756.0){\rule[-0.200pt]{0.400pt}{4.818pt}}
\put(1185.0,131.0){\rule[-0.200pt]{0.400pt}{4.818pt}}
\put(1185,90){\makebox(0,0){ 80}}
\put(1185.0,756.0){\rule[-0.200pt]{0.400pt}{4.818pt}}
\put(1312.0,131.0){\rule[-0.200pt]{0.400pt}{4.818pt}}
\put(1312,90){\makebox(0,0){ 90}}
\put(1312.0,756.0){\rule[-0.200pt]{0.400pt}{4.818pt}}
\put(1439.0,131.0){\rule[-0.200pt]{0.400pt}{4.818pt}}
\put(1439,90){\makebox(0,0){ 100}}
\put(1439.0,756.0){\rule[-0.200pt]{0.400pt}{4.818pt}}
\put(171.0,131.0){\rule[-0.200pt]{0.400pt}{155.380pt}}
\put(171.0,131.0){\rule[-0.200pt]{305.461pt}{0.400pt}}
\put(1439.0,131.0){\rule[-0.200pt]{0.400pt}{155.380pt}}
\put(171.0,776.0){\rule[-0.200pt]{305.461pt}{0.400pt}}
\put(30,453){\rotatebox{-270}{\makebox(0,0){Processing time (ms)}}}
\put(805,29){\makebox(0,0){Payload size (bytes)}}
\put(868.0,131.0){\rule[-0.200pt]{0.400pt}{84.074pt}}
\put(995.0,131.0){\rule[-0.200pt]{0.400pt}{98.287pt}}
\put(1173.0,131.0){\rule[-0.200pt]{0.400pt}{118.041pt}}
\put(1325.0,131.0){\rule[-0.200pt]{0.400pt}{134.904pt}}
\put(1350.0,131.0){\rule[-0.200pt]{0.400pt}{137.795pt}}
\put(1439.0,131.0){\rule[-0.200pt]{0.400pt}{155.380pt}}
\end{picture}
\caption[A GNUplot figure]{Processing time vs. payload length}\vspace{0.5cm}
\label{fig:processing_vs_payload_length}
\end{figure}
\FloatBarrier		

Given these measurements, we can calculate our processing bit rate as the inverse of the time it takes to process an additional byte divided by 8 bits per byte:

\[
	\text{bit rate} = \frac{1}{\frac{\text{time}_{\text{byte}}}{8}} = 20.03 \quad kb/s
\] 

\Cref{tab:majorMarkupLMDetailedResult} shows another table in which some values have been set in bold (using \textbackslash B) to emphasize them. Note how the \texttt{S} formatting has been modified so that it considers the weight of the characters and this is able to decimal align even these hold faced numbers with the numbers in the column above them.

\begin{table}[!ht]
    \centering
    \caption{Median values of sandwich attributes}
    \label{tab:majorMarkupLMDetailedResult}
    \begin{tabular}{l *{2}{S[detect-weight,mode=text,table-format=3.2]}}
        & \multicolumn{2}{c}{\textbf{sites}}\\
        \cmidrule{2-3}
        \textbf{Attribute} & \textbf{A} & \textbf{B} \\
        \midrule
        price (in SEK) & 36.5 & 71.3 \\
        protean (g) & 97.2 & 100.0 \\
        salt (mg) & 9.7 & 9.3 \\
        \hline
        \textbf{Average customer rating in \%} & \B 82.2 & \B 89.9 \\
        \midrule
    \end{tabular}
\end{table}
\FloatBarrier

\section{Reliability Analysis}
\sweExpl{Analys av tillförlitlighet\\
Tillförlitlighet i metod och data}

\section{Validity Analysis}
\sweExpl{Analys av validitet\\
Validitet i metod och data}



% ====================================================================================================
% ====================================================================================================


\cleardoublepage
\chapter{Discussion}
\label{ch:discussion}

\sweExpl{Diskussion\\
Förbättringsförslag?}
\generalExpl{This can be a separate chapter or a section in the previous chapter.}





% ====================================================================================================
% ====================================================================================================

\cleardoublepage
\chapter{Conclusions and Future work}
\label{ch:conclusionsAndFutureWork}

\sweExpl{Slutsats och framtida arbete}

\generalExpl{Add text to introduce the subsections of this chapter.}

\section{Conclusions}
\label{sec:conclusions}
\sweExpl{Slutsatser}
\engExpl{Describe the conclusions (reflect on the whole introduction given in Chapter 1).}


  
\engExpl{Discuss the positive effects and the drawbacks.\\
Describe the evaluation of the results of the degree project.\\
Did you meet your goals?\\
What insights have you gained?\\
What suggestions can you give to others working in this area?\\
If you had it to do again, what would you have done differently?}

\sweExpl{Uppfyllde du dina mål?\\
Vilka insikter har du fått?\\
Vilka förslag kan du ge till andra som arbetar inom detta område?
Om du skulle göra detta igen, vad skulle du ha gjort annorlunda?}

\section{Limitations}
\label{sec:limitations}
\sweExpl{Begränsande faktorer\\Vad gjorde du som begränsade dina ansträngningar? Vilka är begränsningarna i dina resultat?}
\engExpl{What did you find that limited your efforts? What are the limitations of your results?}


\section{Future work}
\label{sec:futureWork}
\sweExpl{Vad du har kvar ogjort?\\Vad är nästa självklara saker som ska göras?\\Vad tips kan du ge till nästa person som kommer att följa upp på ditt arbete?}
\engExpl{Describe valid future work that you or someone else could or should do.\\
Consider: What you have left undone? What are the next obvious things to be done? What hints can you give to the next person who is going to follow up on your work?}



Due to the breadth of the problem, only some of the initial goals have been
met. In these section we will focus on some of the remaining issues that
should be addressed in future work. ...

\subsection{What has been left undone?}
\label{what-has-been-left-undone}

The prototype does not address the third requirment, \ie a yearly unavailability of less than 3 minutes, this remains an open problem. ...

\subsubsection{Cost analysis}
\generalExpl{Example of a missing component}
The current prototype works, but the performance from a cost perspective makes this an impractical solution. Future work must reduce the cost of this solution, to do so a cost analysis needs to first be done. ...

\subsubsection{Security}
\generalExpl{Example of a missing component}
A future research effort is needed to address the security holes that results from using a self-signed certificate. Page filling text mass. Page filling text mass. ...


\subsection{Next obvious things to be done}

In particular, the author of this thesis wishes to point out xxxxxx remains as a problem to be solved. Solving this problem is the next thing that should be done. ...

\section{Reflections}
\label{sec:reflections}
\sweExpl{Reflektioner}
\sweExpl{Vilka är de relevanta ekonomiska, sociala, miljömässiga och etiska aspekter av ditt arbete?}
\engExpl{What are the relevant economic, social,
  environmental, and ethical aspects of your work?
}



One of the most important results is the reduction in the amount of
energy required to process each packet while at the same time reducing the
time required to process each packet.

The thesis contributes to the \gls{UN}\enspace\glspl{SDG} numbers 1 and 9 by
xxxx. 




\noindent\rule{\textwidth}{0.4mm}
\engExpl{In the references, let Zotero or other tool fill this in for you. I suggest an extended version of the IEEE style, to include URLs, DOIs, ISBNs, etc., to make it easier for your reader to find them. This will make life easier for your opponents and examiner. \\IEEE Editorial Style Manual: \url{https://www.ieee.org/content/dam/ieee-org/ieee/web/org/conferences/style_references_manual.pdf}}
\sweExpl{Låt Zotero eller annat verktyg fylla i det här för dig. Jag föreslår en utökad version av IEEE stil - att inkludera webbadresser, DOI, ISBN etc. - för att göra det lättare för läsaren att hitta dem. Detta kommer att göra livet lättare för dina opponenter och examinator.}




% ====================================================================================================
% ====================================================================================================

\cleardoublepage
% Print the bibliography (and make it appear in the table of contents)
\renewcommand{\bibname}{References}
\addcontentsline{toc}{chapter}{References}

\ifbiblatex
    %\typeout{Biblatex current language is \currentlang}
    \printbibliography[heading=bibintoc]
\else
    \bibliography{references}
\fi



\warningExpl{If you do not have an appendix, do not include the \textbackslash cleardoublepage command below, otherwise the last page number in the meta data will be one too large.}
\cleardoublepage
\appendix
\renewcommand{\chaptermark}[1]{\markboth{Appendix \thechapter\relax:\thinspace\relax#1}{}}
\chapter{Something Extra}
\sweExpl{svensk: Extra Material som Bilaga}

\section{Just for testing KTH colors}
\ifdigitaloutput
    \textbf{You have selected to optimize for digital output}
\else
    \textbf{You have selected to optimize for print output}
\fi
\begin{itemize}[noitemsep]
    \item Primary color
    \begin{itemize}
    \item \textcolor{kth-blue}{kth-blue \ifdigitaloutput
    actually Deep sea
    \fi} {\color{kth-blue} \rule{0.3\linewidth}{1mm} }\\

    \item \textcolor{kth-blue80}{kth-blue80} {\color{kth-blue80} \rule{0.3\linewidth}{1mm} }\\
\end{itemize}

\item  Secondary colors
\begin{itemize}[noitemsep]
    \item \textcolor{kth-lightblue}{kth-lightblue \ifdigitaloutput
    actually Stratosphere
    \fi} {\color{kth-lightblue} \rule{0.3\linewidth}{1mm} }\\

    \item \textcolor{kth-lightred}{kth-lightred \ifdigitaloutput
    actually Fluorescence\fi} {\color{kth-lightred} \rule{0.3\linewidth}{1mm} }\\

    \item \textcolor{kth-lightred80}{kth-lightred80} {\color{kth-lightred80} \rule{0.3\linewidth}{1mm} }\\

    \item \textcolor{kth-lightgreen}{kth-lightgreen \ifdigitaloutput
    actually Front-lawn\fi} {\color{kth-lightgreen} \rule{0.3\linewidth}{1mm} }\\

    \item \textcolor{kth-coolgray}{kth-coolgray \ifdigitaloutput
    actually Office\fi} {\color{kth-coolgray} \rule{0.3\linewidth}{1mm} }\\

    \item \textcolor{kth-coolgray80}{kth-coolgray80} {\color{kth-coolgray80} \rule{0.3\linewidth}{1mm} }
\end{itemize}
\end{itemize}

\textcolor{black}{black} {\color{black} \rule{\linewidth}{1mm} }

% Include an example of using nomenclature
\ifnomenclature
\cleardoublepage
\chapter{Main equations}
\label{ch:NomenclatureExamples}
This appendix gives some examples of equations that are used throughout this thesis.
\section{A simple example}
The following example is adapted from Figure 1 of the documentation for the package nomencl (\url{https://ctan.org/pkg/nomencl}).
\begin{equation}\label{eq:mainEq}
a=\frac{N}{A}
\end{equation}
\nomenclature{$a$}{The number of angels per unit area\nomrefeq}%       %% include the equation number in the list
\nomenclature{$N$}{The number of angels per needle point\nomrefpage}%  %% include the page number in the list
\nomenclature{$A$}{The area of the needle point}%
The equation $\sigma = m a$%
\nomenclature{$\sigma$}{The total mass of angels per unit area\nomrefeqpage}%
\nomenclature{$m$}{The mass of one angel}
follows easily from \Cref{eq:mainEq}.

\section{An even simpler example}
The formula for the diameter of a circle is shown in \Cref{eq:secondEq} area of a circle is shown in \cref{eq:thirdEq}.
\begin{equation}\label{eq:secondEq}
D_{circle}=2\pi r
\end{equation}
\nomenclature{$D_{circle}$}{The diameter of a circle\nomrefeqpage}%
\nomenclature{$r$}{The radius of a circle\nomrefeqpage}%

\begin{equation}\label{eq:thirdEq}
A_{circle}=\pi r^2
\end{equation}
\nomenclature{$A_{circle}$}{The area of a circle\nomrefeqpage}%

Some more text that refers to \eqref{eq:thirdEq}.
\fi  %% end of nomenclature example

\cleardoublepage
% Information for authors
%\documentclass[examplethesis.tex]{subfiles}
\begin{document}
\ifxeorlua
\lstdefinestyle{latexExampleForAuthors}{
language=[LaTeX]{TeX},
    breaklines=true,
    postbreak=\mbox{\textcolor{red}{$\hookrightarrow$}\space},
    basicstyle=\small\tt,
    keywordstyle=\color{blue}\sf,
    identifierstyle=\color{magenta},
    commentstyle=\color{cyan},
    backgroundcolor=\color{yellow!15},
    extendedchars=true,
    inputencoding=utf8,
    tabsize=2,
    columns=flexible,
    morekeywords={subtitle, alttitle, altsubtitle, hostcompany, courseCycle,
      courseCode, courseCredits, programcode, degreeName, subjectArea,
      nationalsubjectcategories, todo, ifbiblatex, subsection}
}
\else
\lstdefinestyle{latexExampleForAuthors}{
language=[LaTeX]{TeX},
    breaklines=true,
    postbreak=\mbox{\textcolor{red}{$\hookrightarrow$}\space},
    basicstyle=\small\tt,
    keywordstyle=\color{blue}\sf,
    identifierstyle=\color{magenta},
    commentstyle=\color{cyan},
    backgroundcolor=\color{yellow!15},
    extendedchars=false,
    inputencoding=utf8,
    tabsize=2,
    columns=flexible,
    morekeywords={subtitle, alttitle, altsubtitle, hostcompany, courseCycle,
      courseCode, courseCredits, programcode, degreeName, subjectArea,
      nationalsubjectcategories, todo, ifbiblatex, subsection},
% Support for Swedish, German and Portuguese umlauts
  literate=%
  {Ö}{{\"O}}1
  {Ä}{{\"A}}1
  {Å}{{\AA{}}}1
  {Ü}{{\"U}}1
  {ß}{{\ss}}1
  {ü}{{\"u}}1
  {ö}{{\"o}}1
  {ä}{{\"a}}1
  {å}{{\aa{}}}1
  {á}{{\'a}}1
  {ã}{{\~a}}1
  {é}{{\'e}}1
  {è}{{\`e}}1
  {€}{\euro}1%
  {’}{{\char13}}1
  {-}{{\textendash}}1
  {–}{{\textendash}}1
  {…}{{\ldots}}1,
}
\fi


\chapter{README\_author - the starting place for authors}
\label{ch:READMEauthor}

This document, written by Gerald Q. Maguire Jr, describes the thesis template that I have developed for use at KTH Royal Institute of Technology (KTH). It is important to note that the template is \textbf{not prescriptive}, as not every thesis will have all of the parts that the template shows. However, if there is something that you decide to leave out, you should make a conscious decision to do so and you should consider the impact this may have on your thesis being approved by the examiner.

Fundamental to the design of the template are several key factors:
\begin{itemize}
    \item Helping students be successful in their degree project,
    \item Helping students produce a high-quality thesis, and
    \item Supporting all of the (relevant) phases of the degree project process.
\end{itemize}

\noindent\textbf{This document is a work in progress.}


\section{Advice for Author or Authors}
\label{sec:authors}
One of the hardest problems an author faces is getting started writing, \ie the blank sheet of paper - empty file barrier. The template provides a \mbox{non-blank} starting point; hence, avoiding the blank paper barrier. Additionally, the template provides some initial structure, basically an Introduction, Methods, Results, and Discussion (IMRAD) structure, so that there are hints of were to place material. Moreover, there are places (and notes) about material that the student should consider adding; for example, the ``required reflections'' section in the final chapter.

The template (located in the file \texttt{examplethesis.tex}) also provides some examples of commonly occurring types of content, so that one can easily find examples of how to include a figure, table, code listing, \etc. These examples are not meant to be exhaustive and quite often the student will probably need to learn new \LaTeX\ commands in the course of writing their thesis.

As an author, the first step is to configure the \LaTeX\ engine that you will use to process the files - see \Cref{sec:latexEngine}. The second step will be to configure the template - see \Cref{sec:authorConfigs}. The third step will be to make sure that the information about you, your adviser(s), and examiner are correct in the file \texttt{custom\_configuration.tex} - this information uses the macros described in \Cref{sec:authorMacros}. Now that you have a lot of the administrative details taken care of it is time to start to write - see \Cref{sec:startingToWrite}.

Note that if your are using Overleaf, it is a good idea to rename the project to a name that includes your own name. This will make it easier for your adviser(s) and examiner to find your project in the list of projects that they may have in Overleaf.


If you have more detailed questions about the template itself - 
\iflabelexists{ch:READMEnotes}{see \Cref{ch:READMEnotes}.}
{You have to include the \texttt{README\_notes/README\_notes.tex} file when compiling.}

\section{Author configuration of the \LaTeX\ engine}
\label{sec:latexEngine}
The template should work with pdflatex, XeLaTeX, and LuaLaTeX.  If you are using Overleaf, I strongly recommend the use of XeLaTeX ---  as this will get the \texttt{Arial} fonts correct for the KTH cover. If you are running the compiler on your local machine and you use XeLaTeX \textbf{and} you have \texttt{Arial} as a system font, then it will be able to use it. Similarly, for LuaLaTeX. For pdflatex I have used \textbackslash fontfamily{helvet}, \ie Helvetica, as it is a sans serif font.

One student reported problems with fontspec not loading the fonts properly when running locally with macOS 12.4, TeXLive 2022, LaTeX Workshop on VS Code, and xelatex - the solution is described at \url{https://tug.org/TUGboat/tb39-2/tb122robertson-fontspec.pdf}.


\section{Author configuration of the template}
\label{sec:authorConfigs}
The template is designed to handle a thesis written in English or Swedish.
You can set the default language to `english' or `swedish' by passing an option to the documentclass. Note that the language options is written in all lower case letters; for example, to set the document's language to English:
\begin{lstlisting}[style=latexExampleForAuthors]
\documentclass[english]{kththesis}
\end{lstlisting}

To set the document's language to Swedish:
\begin{lstlisting}[style=latexExampleForAuthors]
\documentclass[swedish]{kththesis}
\end{lstlisting}

The language option `swedish' sets the conditional \texttt{\textbackslash ifinswedish} to true.  Among many other things, this conditional is used to configure the KTH cover and the title page to use the chosen language.

The two most common bibliographic engines are supported, \ie BibTeX and BibLaTeX. To set the language to English and use the bibliographic engine to BibTeX you would say:
\begin{lstlisting}[style=latexExampleForAuthors]
\documentclass[english, bibtex]{kththesis}
\end{lstlisting}
To set the language to Swedish and use the bibliographic engine to BibLaTeX you would say:
\begin{lstlisting}[style=latexExampleForAuthors]
\documentclass[swedish, biblatex]{kththesis}
\end{lstlisting}

The above illustrates that you can pass multiple options to the document class separated by commas. Also note that the options were passed as all lower case letters.

You can of course also modify the formatting of the citations and bibliography. See for example the following code snippet:

\begin{lstlisting}[style=latexExampleForAuthors]
\ifbiblatex
    %\usepackage[language=english,bibstyle=authoryear, citestyle=authoryear, maxbibnames=99]{biblatex}
    %\usepackage[style=numeric,sorting=none,backend=biber]{biblatex}
    \usepackage[bibstyle=authoryear,citestyle=authoryear, maxbibnames=99,language=english]{biblatex}
    % alternatively you might use another style, such as IEEE
    %\usepackage[style=ieee]{biblatex}
    \addbibresource{references.bib}
    %\DeclareLanguageMapping{norsk}{norwegian}
\else
    % The line(s) below are for BibTeX
    \bibliographystyle{bibstyle/myIEEEtran}
    %\bibliographystyle{apalike}
\fi
\end{lstlisting}

To optimize for digital output (this changes the color palette) add the option: \texttt{digitaloutput}. There are also options for A4 or G6 paper: \texttt{a4paper} or \texttt{g5paper} (respectively). The is an option for \texttt{nomenclature}, to produce and refer to equations
\ifnomenclature
as shown in \Cref{ch:NomenclatureExamples}
\fi
.  Finally, there are options for a 1\textsuperscript{st} cycle thesis or 2\textsuperscript{nd} cycle thesis: \texttt{bachelor} and \texttt{master} (respectively); however, these two options are \textbf{not} currently used.

One of the first things that the author(s) will want to do is add the working title and subtitle to the thesis. This is done using the \textbackslash title, \textbackslash subtitle, \textbackslash alttitle, and \textbackslash altsubtitle macros as shown below:
\begin{lstlisting}[style=latexExampleForAuthors]
\title{This is the title in the language of the thesis}
\subtitle{A subtitle in the language of the thesis}

% give the alternative title - i.e., if the thesis is in English,
% then give a Swedish title
\alttitle{Detta är den svenska översättningen av titeln}
\altsubtitle{Detta är den svenska översättningen av undertiteln}
% alternative, if the thesis is in Swedish, then give an English title
%\alttitle{This is the English translation of the title}
%\altsubtitle{This is the English translation of the subtitle}   
\end{lstlisting}

Setting these values once and then using them in many places reduces the work to change them while at the same time ensuring consistency. 

Some additional configuration that the author(s) might do is to set the values of the macros related to the course cycle, course code, date of the thesis, number of credits, degree/exam name, subject area, and if the degree is done external to KTH to set the host information. Consider the snippet below for a student admitted to the ``Bachelor's Programme in Information and Communication Technology (TCOMK)'' program and enrolled in the degree project course ``IA150X Degree Project in Information and Communication Technology, First Cycle 15.0 credits'' and working at a company ``Företaget AB'':
\begin{lstlisting}[style=latexExampleForAuthors]
\hostcompany{Företaget AB} % Remove this line if the project was not done at a host company

\date{\today}

\courseCycle{1}
\courseCode{IA150X}
\courseCredits{15.0}

\programcode{TCOMK}
\degreeName{Bachelors degree}
% Note that the subject area for a Bachelor's thesis (Kandidatexamen)
% should be either Technology or Architecture
% If the thesis is in Swedish, these would be: teknik | arkitektur
% -- Note the use of lower case for the Swedish subject area
\subjectArea{Technology}
\end{lstlisting}

Note that in the above macros you have to give the English or Swedish names in the arguments to \textbackslash degreeName and \textbackslash subjectArea - as shown below:
\begin{lstlisting}[style=latexExampleForAuthors]
\degreeName{Kandidatexamen}
\subjectArea{teknik}
\end{lstlisting}

For a CDATE student enrolled in the course ``DA231X Degree Project in Computer Science and Engineering, Second Cycle 30.0 credits'', the cycle, program, course code, degree, and subject area information would be:
\begin{lstlisting}[style=latexExampleForAuthors]
\programcode{CDATE}
\courseCycle{2}
\courseCode{DA231X}
\courseCredits{30.0}
\degreeName{Degree of Master of Science in Engineering}
\subjectArea{Computer Science and Engineering}
\end{lstlisting}

You can find a list of the program codes and school acronyms in the file:\linebreak[4] \texttt{lib/schools\_and\_programs.ins}.

There are a set of rules about what is to be displayed on the KTH cover. These can be found at \url{https://www.kth.se/social/group/sprakkommitten/page/omrade-for-examensarbete/}.

One of the reasons for many of the macros shown above and below are to collect the information that is needed to report the approved thesis in Digitala Vetenskapliga Arkivet (DiVA) and to report the title(s) and grade in Lokalt adb–baserat dokumentationssystem (LADOK).

National subject categories are a required field in the DiVA record. These categories follow a definition by SCB and HSV.
While these code refer to research areas, these codes are also used in KTH to indicate the area of the thesis. The guidance that I received from the Linköping library was that one should try to use 5 digit codes when possible. Some examples of these codes are shown in Table~\ref{tab:nationalsubject categories}.
\begin{description}[leftmargin=!, labelwidth =\widthof{\texttt{\textbackslash nationalsubjectcategories\{\}}}]
\item [\texttt{\textbackslash nationalsubjectcategories\{\}}] comma separated list of national subject category codes - each a 3 or 5 digit code
\end{description}

An example for a thesis in Computer Science and Computer Systems:
\begin{lstlisting}[style=latexExampleForAuthors]
\nationalsubjectcategories{10201, 10206}
\end{lstlisting}

You can find the subjects and their codes in:\\ \url{https://www.scb.se/contentassets/3a12f556522d4bdc887c4838a37c7ec7/standard-for-svensk-indelning--av-forskningsamnen-2011-uppdaterad-aug-2016.pdf}\\
and\\
\url{https://www.scb.se/contentassets/10054f2ef27c437884e8cde0d38b9cc4/oversattningsnyckel-forskningsamnen.pdf}

\begin{table}[!ht]
  \begin{center}
    \caption{Examples of some national subject categories and their codes}
    \label{tab:nationalsubject categories}
    \begin{tabular}{p{0.85cm} L{6.2cm} L{6.2cm}} % <-- Alignments: 1st column left, 2nd middle, with vertical lines in between
      \textbf{Code}  & \textbf{Category (in Swedish)} & \textbf{Category (in English)} \\
      \hline
102 & Data- och informationsvetenskap (Datateknik) &   Computer and Information Sciences \\
      \hline
10201 & Datavetenskap (datalogi) & Computer Sciences \\
      \hline
10202 & Systemvetenskap, informationssystem och informatik (samhällsvetenskaplig inriktning under 50804) &
Information Systems (Social aspects to be 50804)\\
      \hline
10203 & Bioinformatik (beräkningsbiologi) (tillämpningar under 10610) & Bioinformatics (Computational Biology) (applications to be 10610) \\
      \hline
10204 & Människa-datorinteraktion (interaktionsdesign) (Samhällsvetenskapliga aspekter under 50803) & Human Computer Interaction (Social aspects to be 50803)\\
      \hline
10205 & Programvaruteknik & Software Engineering \\
      \hline
10206 & Datorteknik & Computer Engineering \\
      \hline
10207 & Datorseende och robotik (autonoma system) & Computer Vision and Robotics (Autonomous Systems) \\
      \hline
10208 & Språkteknologi (språkvetenskaplig databehandling) & Language Technology (Computational Linguistics) \\
      \hline
10209 & Medieteknik & Media and Communication Technology \\
      \hline
10299 & Annan data- och informationsvetenskap & Other Computer and Information Science \\
      \hline
      \hline
202   & Elektroteknik och elektronik & Electrical Engineering, Electronic Engineering, Information Engineering \\
      \hline
20201 & Robotteknik och automation & Robotics \\
      \hline
20202 & Reglerteknik & Control Engineering \\
      \hline
20203 & Kommunikationssystem & Communication Systems \\
      \hline
20204 & Telekommunikation & Telecommunications \\
      \hline
20205 & Signalbehandling & Signal Processing \\
      \hline
20206 & Datorsystem & Computer Systems \\
      \hline
20207 & Inbäddad systemteknik & Embedded Systems \\
      \hline
20299 & Annan elektroteknik och elektronik & Other Electrical Engineering, Electronic Engineering, Information Engineering \\
      \hline
    \end{tabular}
  \end{center}
\end{table}


\FloatBarrier



\section{Author macros}
\label{sec:authorMacros}
It is assumed that there can only be 1 or 2 authors. For many years now 2\textsuperscript{nd} cycle theses are expected to only have one author.

For the author or first author, there are a number of macros defined to store information about the author, so that it can later be used in multiple places -- for example, the KTH cover (produced with \texttt{\textbackslash kthcover)}, the title page (produced with \texttt{\textbackslash titlepage}, the ``For DIVA'' section at the end of the thesis (produced with \linebreak[4]
\texttt{\textbackslash divainfo\{pg:lastPageofPreface\}\{pg:lastPageofMainmatter\}}), and possibly a JavaScript Object Notation (JSON) file named \texttt{fordiva.json} produced as a by product of the \texttt{\textbackslash divainfo}. Note that the actual section name has DiVA set in all caps - which hopefully should not occur in the thesis! If the string DiVA set in all caps, does have to appear, then the section heading should be preceded by four euro signs and followed by four more euro signs (as is done this this doucment).

The author related macros are:
\begin{description}[leftmargin=!, labelwidth =\widthof{\texttt{\textbackslash secondAuthorsFirstname\{\}}}]
\item [\texttt{\textbackslash authorsLastname\{\}}] the last name of the author

\item [\texttt{\textbackslash authorsFirstname\{\}}] the first name of the author

\item [\texttt{\textbackslash email\{\}}] the KTH e-mail address of the author

\item [\texttt{\textbackslash kthid\{\}}] the author's kthid, this generally start with the string ·``u1'' and is a unique identifier for every KTH user.

% As per email from KTH Biblioteket on 2021-06-28 students cannot have an OrCiD reported for their degree project
\item [\texttt{\textbackslash authorsSchool\{\}}] the value is generally of the form:\linebreak[4] \texttt{\textbackslash schoolAcronym\{EECS\}}. The currently supported school acronyms are: ABE, CBH, EECS, ITM, and SCI. These are defined in the file\linebreak[4] \texttt{schools\_and\_programs.ins}.
\end{description}

If the first author is not in Stockholm, Sweden when the acknowledgements are written, then add that information via the macros described below.
This information will be used when generating the acknowledgements signature. The acknowledgements signature is the text at the end of the acknowledgements and it gives the place where the author(s) is/are when writing the acknowledgements and also gives the date and name(s).
\begin{description}[leftmargin=!, labelwidth =\widthof{\texttt{\textbackslash secondAuthorsFirstname\{\}}}]
\item [\texttt{\textbackslash authorCity\{A City\}}] specify the city

\item [\texttt{\textbackslash authorCountry\{A Country\}}] specify the country

\item [\texttt{\textbackslash authorCityCountryDate\{\}}] pass into this function the month and year for the acknowledgement. This can be a string such as January 2022 or it can be a \LaTeX\ expression, such as \textbackslash MONTH\textbackslash enspace\textbackslash the\textbackslash year.
\end{description}

If there is a second author and the place, month, and year are \textbf{all} the same, then specify the month and year for only the \textbf{first} author:
\begin{lstlisting}[style=latexExampleForAuthors]
\authorCityCountryDate{\MONTH\enspace\the\year}
\end{lstlisting}

If there is a second author and the place is different, then say:
\begin{lstlisting}[style=latexExampleForAuthors]
\authorCityCountryDate{}
\end{lstlisting}
\clearpage

If there is a second author, the macros are:
\begin{description}[leftmargin=!, labelwidth =\widthof{\texttt{\textbackslash secondAuthorsFirstname\{\}}}]
\item [\texttt{\textbackslash secondAuthorsLastname\{\}}] the last name of the 2\textsuperscript{nd} author
\item [\texttt{\textbackslash secondAuthorsFirstname\{\}}] the first name of the 2\textsuperscript{nd} author
\item [\texttt{\textbackslash secondemail\{\}}] the KTH e-mail address of the 2\textsuperscript{nd} author
\item [\texttt{\textbackslash secondkthid\{\}}] the 2\textsuperscript{nd} author's kthid
% As per email from KTH Biblioteket on 2021-06-28 students cannot have an OrCiD reported for their degree project
\item [\texttt{\textbackslash secondAuthorsSchool\{\}}] the school of the 2\textsuperscript{nd} author
\end{description}

If the second author is not in the same place as the first author, then add the relevant information using the macros below.  This information will be used when generating the acknowledgements signature.
\begin{description}[leftmargin=!, labelwidth =\widthof{\texttt{\textbackslash secondAuthorsFirstname\{\}}}]
\item [\texttt{\textbackslash secondAuthorCity\{A City\}}]  specify the city

\item [\texttt{\textbackslash secondAuthorCountry\{A Country\}}] specify the country

\item [\texttt{\textbackslash secondAuthorCityCountryDate\{\textbackslash MONTH\textbackslash enspace\textbackslash the\textbackslash year\}}]  pass into this function the month and year for the acknowledgement
\end{description}

If the second author is the same place as the first author, then comment out or delete the \textbackslash secondAuthorCityCountryDate\{\} as shown below:
\begin{lstlisting}[style=latexExampleForAuthors]
%\secondAuthorCityCountryDate{}
\end{lstlisting}

\section{Starting to write}
\label{sec:startingToWrite}

As you write you will notice "todo" notes in the template. They follow the following conventions:
\begin{lstlisting}[style=latexExampleForAuthors]
\generalExpl{Comments/directions/... in English}
\sweExpl{Text på svenska}
\engExpl{English descriptions about formatting}
\sweExpl{warnings}
\end{lstlisting}


\subsection{Working abstract}
\label{sec:wrtingFirstAbstract}
I generally recommend that every student start by writing a working abstract, this will help you keep your focus. To find where you can start to enter your abstract look in the \textit{examplethesis.tex} file for the line:
\begin{lstlisting}[style=latexExampleForAuthors]
\generalExpl{Enter your abstract here!}
\end{lstlisting}

There is lots of information already in the template to help you with entering text, equations, \etc in your abstract.

\subsection{Acronyms}
\label{sec:addingAcronyms}
You may want to define an acronym to help you with your writing, as this can both reduce the amount of typing and help your reader by providing consistent use of acronyms. The acronyms definitions can be found in the file \textit{lib/acronyms.tex}. The file contains some examples. I generally try to sort the lines to help find which acronyms I already have defined and keep track of the new one(s) I want to add.

\subsection{Some predefined macros to help when writing}
\label{sec:predefine}

The file \textit{lib/defines.tex} includes some macros that will help you when writing. This includes \textbackslash etc, to give you ``\etc'', \textbackslash eg, \textbackslash ie, and \textbackslash etal.
The file also defined \textbackslash first, \textbackslash Second, ... \textbackslash eighth to give you \first, \Second, \third, ... \eighth. Note that `Second' is written with an initial capital letter to avoid conflict with the unit `second' in the \texttt{siunitx} package.

\subsection{Additional abstract(s)}
\label{sec:additionalAbstracts}

All theses at KTH are \textbf{required} to have an abstract in both \textit{English} and \textit{Swedish}. However, in addition to this there are many students who want to add abstracts in additional languages. The template comes pre-configured with places for abstracts in a number of other languages. If there is a language that you want to use that is not already supported there are directions for how to add an additional language. If there are abstracts in languages that you do not want, please delete them or comment them out (see \Cref{sec:hideComment}).

I suggest avoiding the use of the defined acronyms in abstracts other than the English abstract. This is due to the fact that the \texttt{glossaries} package (that is being used to support acronyms) does not directly provide support for multiple languages and because I do not understand how to programmatically create plurals of acronyms in Swedish or other languages.

\subsection{Removing and hiding parts that you do not want}
\label{sec:hideComment}

It is quite likely that you will find parts of the template that you do not want/need. One way of dealing with this is to delete them and other way is to comment them out. Personally, I like to comment things out, in case I actually do want to be able to read it in the \LaTeX\null file or uncomment it later. To comment out a portion of the file simply use the following environment:

\begin{lstlisting}[style=latexExampleForAuthors]
\begin{comment}
    **** what you want to comment out ****
\end{comment}
\end{lstlisting}

For example, if you are not interested in the Swedish language \texttt{todo} notes, you can look for lines with ``backgroundcolor=kth-lightblue40'' in them and comment them out (or delete them).

\subsection{Removing the README\_notes}
At some point you will no longer want this README information. You can remove it by removing the line
\textbackslash include\{README\_notes/README\_notes\} -- from the \textit{examplethesis.tex} file. You can then remove the \textbf{README\_notes} directory.

Unless you are an examiner or an administrator you can delete the file: \texttt{README\_notes/README\_examiner\_notes.tex} and delete the include of this file from near the end of the template (\ie \textit{examplethesis.tex}. You can also delete the directory \textbf{README\_notes/README\_examiner-figures}.


\section[Copyright or Creative Commons License]{Copyright or Creative Commons\\ License}
\label{sec:copyrightOrCClicense}
It is possible to have several variants of the bookinfo page:
\begin{enumerate}[labelwidth =\widthof{\textbf{Creative Commons (CC)}}, leftmargin = !]
    \item[copyright] If you want to have a bookinfo page, include the line saying \textbackslash bookinfopage.
    \item[Creative Commons (CC)] If you want to have a bookinfo page but want to have a Creative Commons license, then include \textbackslash bookinfopage and use and configure the \texttt{doclicense} package as described below.
    \item[none] If you do \textbf{not} want to have a bookinfo page, comment the line saying \textbackslash bookinfopage and add a \textbackslash cleardoublepage.
\end{enumerate}

For background about Creative Commons licenses see:
\url{https://www.kb.se/samverkan-och-utveckling/oppen-tillgang-och-bibsamkonsortiet/open-access-and-bibsam-consortium/open-access/creative-commons-faq-for-researchers.html} and \url{https://kib.ki.se/en/publish-analyse/publish-your-article-open-access/open-licence-your-publication-cc}.

Note that the lowercase version of the Creative Commons license has to be used in the modifier, \ie one of: by, by-nc, by-nd, by-nc-nd, by-sa, by-nc-sa, or zero. For the list of supported licenses see the documentation for the \texttt{doclicense} package.

Note that if the \texttt{doclicense} package is used it automatically redefines \texttt{\textbackslash bookinfopage} to be \texttt{\textbackslash bookinfopageCC}.

\subsection{Example configuration to have a CC BY-NC-ND license}

\begin{lstlisting}[style=latexExampleForAuthors]
\usepackage[
    type={CC},
    modifier={by-nc-nd},
    version={4.0},
    hyphenation={RaggedRight},
]{doclicense}
\end{lstlisting}

Note that the option ``hyphenation={RaggedRight}'' can be used with the configuration of the package to set the license information with a ragged right margin rather that as a fill and justified paragraph.


\subsection{Example configuration to have a CC BY-NC-ND license with a Euro symbol rather than a Dollar sign}

\begin{lstlisting}[style=latexExampleForAuthors]
\usepackage[
    type={CC},
    modifier={by-nc-nd},
    version={4.0},
    imagemodifier={-eu-88x31},  % to get Euro symbol rather than Dollar sign
    hyphenation={RaggedRight},
]{doclicense}
\end{lstlisting}


\subsection{Example configuration to have a CC0 license}

\begin{lstlisting}[style=latexExampleForAuthors]
\usepackage[
    type={CC},
    modifier={zero},
    version={1.0},
]{doclicense}
\end{lstlisting}



\end{document}

\subfile{README_author}

\cleardoublepage
% information about the template for everyone
% \input{README_notes/README_notes}

\begin{comment}
% information for examiners
\ifxeorlua
\cleardoublepage
\input{README_notes/README_examiner_notes}
\fi
\end{comment}

\begin{comment}
% information for administrators
\ifxeorlua
\cleardoublepage
\input{README_notes/README_for_administrators.tex}
\fi
\end{comment}

%% The following label is necessary for computing the last page number of the body of the report to include in the "For DIVA" information
\label{pg:lastPageofMainmatter}

\cleardoublepage
\clearpage\thispagestyle{empty}\mbox{} % empty page with backcover on the other side
\kthbackcover
\fancyhead{}  % Do not use header on this extra page or pages
\section*{€€€€ For DIVA €€€€}
\lstset{numbers=none} %% remove any list line numbering
\divainfo{pg:lastPageofPreface}{pg:lastPageofMainmatter}

% If there is an acronyms.tex file,
% add it to the end of the For DIVA information
% so that it can be used with the abstracts
\IfFileExists{lib/acronyms.tex}{
\section*{acronyms.tex}
\lstinputlisting[language={[LaTeX]TeX}, basicstyle=\ttfamily\color{black},
commentstyle=\color{black}, backgroundcolor=\color{white}]{lib/acronyms.tex}
}
{}

\end{document}
